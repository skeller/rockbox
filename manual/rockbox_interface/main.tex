% $Id$ %
\chapter{\label{ref:rockbox_interface}Quick Start}
\section{Basic Overview}
\subsection{The \daps{} controls}

\begin{center}
% include the front image. Using \specimg makes this fairly easy,
% but requires to use the exact value of \specimg in the filename!
% The extension is selected in the preamble, so no further \ifpdfoutput
% is necessary.
\includegraphics[height=8cm,width=10cm,keepaspectratio=true]{rockbox_interface/images/\specimg-front}
\opt{iaudiom3,iriverh100}{% replace with HAVEREMOTEKEYMAP when all images exist or change specimg
  \end{center}
  % spacing between the two pictures, could possibly be improved
  \begin{center}
    \includegraphics[height=5.6cm,width=10cm,keepaspectratio=true]{rockbox_interface/images/\specimg-remote}
}
\end{center}

Throughout this manual, the buttons on the \dap{} are labelled according to the
picture above.
\opt{touchscreen}{
The areas of the touchscreen in the 3$\times$3 grid mode are in turn referred as follows:
\begin{table}
    \centering
    \begin{tabular}{|c|c|c|}
	\hline
        \TouchTopLeft & \TouchTopMiddle & \TouchTopRight \\ [5ex]
	\hline        
	\TouchMidLeft & \TouchCenter & \TouchMidRight \\ [5ex]
	\hline        
	\TouchBottomLeft & \TouchBottomMiddle & \TouchBottomRight \\ [5ex]
	\hline
    \end{tabular}
\end{table}
}%
Whenever a button name is prefixed by ``Long'', a long press of approximately
one second should be performed on that button. The buttons are described in
detail in the following paragraph.
\blind{%
  Additional information for blind users is available on the Rockbox website at 
  \wikilink{BlindFAQ}.
  
  %
  \opt{iriverh100}{
  Hold or lay the \dap{} so that the side with the joystick and LCD is facing
  towards you, and the curved side is at the top. The joystick functions as
  the \ButtonUp{}, \ButtonRight{}, \ButtonLeft{}, and \ButtonDown{} buttons when
  pressed in the appropriate direction. Pressing the joystick down functions as
  \ButtonSelect{}. 
  On the right side of the \dap{} are the \ButtonOn{}, \ButtonOff{}, 
  \ButtonMode{} buttons, and the \ButtonHold{} switch. When this switch is
  switched towards the bottom of the \dap{}, hold is on, and none of the other
  buttons have any effect.

  On the left side is the \ButtonRec{} button. Above that is the internal microphone. 

  On the top panel of the \dap{}, from left to right, you can find the
  following: headphone mini jack plug, remote port, Optical line-in, Optical line-out.

  On the bottom panel of the \dap{}, from left to right, you can find the
  following: power jack, reset switch, and USB port. In the event that your
  \dap{} hard locks, you can reset it by inserting a paper clip into the hole
  where the reset switch is.}
  % 
  \opt{iriverh300}{
  Hold or lay the \dap{} so that the side with the button pad and
  LCD is facing towards you.  The buttons on the button pad are as follows:  top 
  left corner: \ButtonOn{}, bottom left corner: \ButtonOff{}, top right corner: 
  \ButtonRec, bottom right corner: \ButtonMode{}.  In the center of the button pad 
  is a button labelled \ButtonSelect{}.  Surrounding the \ButtonSelect{} button are
  the \ButtonUp{}, \ButtonDown{}, \ButtonLeft{}, and \ButtonRight{} buttons.
  
  On the top panel of the \dap{}, from left to right, you can find the 
  following: headphone mini jack plug, remote port, line-in, line-out.

  On the left hand side of the \dap{} is the internal microphone. Just underneath
  this is a small hole, the reset switch. In the event that your \dap{} hard locks,
  you can reset it by inserting a paper clip into the hole where the reset switch
  is.

  On the right hand side of the \dap{} is the \ButtonHold{} switch. When this is 
  switched towards the bottom of the \dap{}, hold is on, and none of the other 
  buttons have any effect.
  
  On the bottom panel of the \dap{}, from left to right, you can find the 
  following:  power jack and two USB ports.  The USB port on the right is used 
  to connect your \dap{} to your computer.  The USB port on the left is not 
  used in Rockbox. 
  }
  %
  \opt{mpiohd200}{
  Hold or lay the \dap{} so that the side with the LCD is facing towards you.
  On the right hand side there is a rocker switch at the top which serves as
  \ButtonRew{} and \ButtonFF{} when rocked up or down, respectively.
  Pressing the rocker in functions as the \ButtonFunc{} button. Below the rocker 
  there are the \ButtonRec{} and \ButtonPlay{} buttons. At the bottom of the 
  right panel there is the \ButtonHold{} switch. When this is switched towards the
  bottom of the \dap{}. hold is on, and none of the other buttons have any effect.

  On the top panel of the \dap{} there is another rocker which serves as the 
  \ButtonVolDown{} and \ButtonVolUp{} buttons when pressed to the left or right,
  respectively.

  On the left hand side of the \dap{} there is a headphone mini jack plug at the top
  and a small hole at the bottom, the reset switch. In the event that your \dap{}
  hard locks, you can reset it by inserting a paper clip into the hole where the 
  reset switch is.

  On the bottom panel of the \dap{}, from left to right, you can find the
  following: power jack, line-in jack and USB port (under rubber cover).
  }
  %
  \opt{ipod4g,ipodcolor,ipodvideo,ipodmini}{ 
  The main controls on the \dap{} are a slightly indented scroll wheel 
  with a flat round button in the center. Hold the \dap{} with these controls 
  facing you. 

  The top of the player will have the following, from left to 
  right:
  \opt{ipod4g,ipodcolor}{remote connector, headphone socket, \ButtonHold{} 
    switch.}
  \opt{ipodvideo}{\ButtonHold{} switch, headphone socket.}
  \opt{ipodmini}{\ButtonHold{} switch, remote connector, headphone socket.}	

  The dock connector that is used to connect your \dap{} to your computer is on 
  the bottom panel of the \dap{}.

  The button in the middle of the wheel is called \ButtonSelect{}. You can
  operate the wheel by pressing the top, bottom, left or right sections,
  or by sliding your finger around it.  The top is \ButtonMenu{}, the bottom is
  \ButtonPlay{}, the left is \ButtonLeft{}, and the right is \ButtonRight{}.
  When the manual says to \ButtonScrollFwd{}, it means to slide your finger
  clockwise around the wheel. \ButtonScrollBack{} means to slide your finger
  counterclockwise. Note that the wheel is sensitive, so you will need to move
  slowly at first and get a feel for how it works.
  
  Note that when the \ButtonHold{} switch is pushed toward the center of the \dap{}, 
  hold is on, and none of the other controls do anything.  Be sure
  \ButtonHold{} is off before trying to use your player. 
  }
  %
  \opt{ipod3g}{ 
  The main controls on the \dap{} are a slightly indented touch wheel 
  with a flat round button in the center, and four buttons in a row above the
  touch wheel. Hold the \dap{} with these controls 
  facing you. 

  The top of the player will have the following, from left to 
  right: remote connector, headphone socket, \ButtonHold{} switch.
	
  The dock connector that is used to connect your \dap{} to your computer is on 
  the bottom panel of the \dap{}.

  The button in the middle of the wheel is called \ButtonSelect{}. You can
  operate the wheel by sliding your finger around it.  The row of
  buttons consists of, from left to right, the \ButtonLeft{},
  \ButtonMenu{}, \ButtonPlay{}, and \ButtonRight{} buttons.
  When the manual says to \ButtonScrollFwd{}, it means to slide your finger
  clockwise around the wheel. \ButtonScrollBack{} means to slide your finger
  counterclockwise. Note that the wheel is sensitive, so you will need to move
  slowly at first and get a feel for how it works.
  
  Note that when the \ButtonHold{} switch is pushed toward the center of the \dap{}, 
  hold is on, and none of the other controls do anything.  Be sure
  \ButtonHold{} is off before trying to use your player. 
  }
  %
  \opt{ipod1g2g}{ 
  The main controls on the \dap{} are a slightly indented wheel 
  with a flat round button in the center, and four buttons surrounding
  it. On the 1st generation iPod, this wheel physically turns. On the
  2nd generation iPod, this wheel is touch-sensitive. Hold the \dap{} with these controls 
  facing you. 

  The top of the player will have the following, from left to 
  right: FireWire port, headphone socket, \ButtonHold{} switch.

  The FireWire port is used to connect your \dap{} to the computer and
  to charge its battery via a wall charger.
	
  The button in the middle of the wheel is called \ButtonSelect{}. You can
  operate the wheel by turning it, or sliding your finger around
  it. The top is \ButtonMenu{}, the bottom is \ButtonPlay{}, the left
  is \ButtonLeft{}, and the right is \ButtonRight{}.
  When the manual says to \ButtonScrollFwd{}, it means to slide your finger
  clockwise around the wheel. \ButtonScrollBack{} means to slide your finger
  counterclockwise. Note that the wheel is sensitive, so you will need to move
  slowly at first and get a feel for how it works.
  
  Note that when the \ButtonHold{} switch is pushed toward the center of the \dap{}, 
  hold is on, and none of the other controls do anything.  Be sure
  \ButtonHold{} is off before trying to use your player. 
  }
  %
  \opt{ipodnano,ipodnano2g}{
  The main controls on the \dap{} are a slightly indented wheel with a
  flat round button in the center. Hold the \dap{} with these controls on the
  top surface. There is a \ButtonHold{} switch at one end, and
  headphone and dock connector at the other; be sure the end with the
  switch is facing away from you.

  The button in the middle of the wheel is called \ButtonSelect{}. You can
  operate the wheel by pressing the top, bottom, left or right sections,
  or by sliding your finger around it.  The top is \ButtonMenu{}, the bottom is
  \ButtonPlay{}, the left is \ButtonLeft{}, and the right is \ButtonRight{}.
  When the manual says to \ButtonScrollFwd{}, it means to slide your finger
  clockwise around the wheel. \ButtonScrollBack{} means to slide your finger
  counterclockwise. Note that the wheel is sensitive, so you will need to move
  slowly at first and get a feel for how it works.

  Note that when the \ButtonHold{} switch is pushed toward the center of the \dap{},
  hold is on, and none of the other controls do anything; be sure \ButtonHold{} is
  off before trying to use your player.
  }
  %
  \opt{ondio}{
  The main characteristic of the Ondio case is the indent on its lower right side, 
  which is the MMC slot. Holding the \dap{} with this slot in the described position
  you'll find the following:

  On the curved top, from left to right, are the headphone socket,
  the \ButtonOff{} button,%
  \opt{recording}{ and the line-in jack}.
  Apart from the already mentioned MMC slot, you will find the USB connector on
  the \daps{} right side. Below the LCD, at approximately the center of the \dap{},
  there is the main button pad of the \dap{}. The centre of the button pad dips inward
  and helps to operate the directional keys from there. Located on a two-way button
  strip are the \ButtonLeft{} and \ButtonRight{} keys, with \ButtonUp{} above it
  and \ButtonDown{} below it. The raised button positioned in the lower left of this 
  round crosspad is labelled \ButtonMenu{}.
  }
  %
  \opt{iriverh10,iriverh10_5gb}{
  Hold or lay the \dap{} so that the side with the scroll pad and
  LCD is facing towards you. In the centre below the lcd is the scroll pad. It
  is oriented vertically. Touching the top and bottom half of it acts as the 
  \ButtonScrollUp{}  and \ButtonScrollDown{} buttons respectively. On the left
  of the scroll pad is the \ButtonLeft{} button and on the right is the
  \ButtonRight{} button.
  
  There are three buttons on the right hand side of the \dap{}. From top to 
  bottom, they are: \ButtonRew{}, \ButtonPlay{} and \ButtonFF{}. On the left 
  hand side is the \ButtonPower{} button.

  On the top panel of the \dap{}, from left to right, you can find the 
  following: \ButtonHold{} switch, \opt{iriverh10}{reset pin hole, }remote port
  and headphone mini jack plug. 
  
  On the bottom panel of the \dap{} is the data cable port.}
  %
  \opt{gigabeatf}{
  \note{The following description is for the Gigabeat F, but can also apply for the
  Gigabeat X. The Gigabeat F is slightly larger and more rectangular shaped, while the
  Gigabeat X is smaller and has a slightly tapered back.}

  Hold the \dap{} with the screen on top and the controls on the right hand side.  
  Below the screen is a cross-shaped touch sensitive pad which contains the 
  \ButtonUp{}, \ButtonDown{}, \ButtonLeft{} and \ButtonRight{} controls.  On the
  Gigabeat X, this pad will feel slightly raised up, while it will feel slightly
  sunken in on the Gigabeat F. On the top of the unit, from left to right, are the 
  power socket, the \ButtonHold{} switch, and the headphone socket.  The 
  \ButtonHold{} switch puts the \dap{} into hold mode when it is switched to the 
  right of the unit. The buttons will have no effect when this is the case.  
  
  Starting from the left hand side on the bottom of the unit, nearer to the front
  than the back, is a recessed switch which 
  controls whether the battery is on or off.  When this switch is to the left,
  the battery is disconnected.  This can be used for a hard reset of the unit,
  or if the \dap{} is being placed in storage.  Next to that is a connector for
  the docking station and finally on the right hand side of the bottom of the
  unit is a mini USB socket for connecting directly to USB.
  
  Finally on the right hand side of the unit are some control buttons.  Going from
  the bottom of the unit to the top there is a small round \ButtonA{} buttton then a
  rocker volume switch with of the \ButtonVolDown{} button below the \ButtonVolUp{}
  button.  Above that is are two more small round buttons, the \ButtonMenu{} 
  button and nearest to the top of the unit the \ButtonPower{} button, which is held
  down to turn the \dap{} on or off. If you have a Gigabeat X, these buttons are small
  metallic buttons that are place further up on the right hand side, and closer
  together. The layout is still the same, however.}
  %
  \opt{gigabeats}{
  Hold the \dap{} with the screen on top and the controls on the right hand side.
  Directly below the bottom edge of the screen are two buttons, \ButtonBack{}
  on the left and \ButtonMenu{} on the right. Below them is a cross-shaped pad
  which contains the \ButtonUp{}, \ButtonDown{}, \ButtonLeft{}, \ButtonRight{}
  and \ButtonSelect{} controls.
  On the top of the unit from left to right are the headphone socket and the
  \ButtonHold{} switch.  The \ButtonHold{} switch puts the \dap{} into
  hold mode when it is switched to the right of the unit.
  The buttons will have no effect when this is the case.

  Starting from the left hand side on the bottom of the unit, nearer to the back
  than the front, is a recessed switch which controls whether the battery is on
  or off.  When this switch is to the left, the battery is disconnected.
  This can be used for a hard reset of the unit, or if the \dap{} is being placed
  in storage.  Next to that is a mini USB socket for connecting directly to USB, 
  and finally a custom connector, presumably for planned accessories which were 
  never released.

  Finally on the right hand side of the unit are some control buttons and the power 
  connector.  Going from the bottom of the unit to the top, there is the power 
  connector socket, followed by three small round buttons, the
  \ButtonNext{} buttton, \ButtonPlay{} button, and \ButtonPrev{} button (from bottom
  to top) then a rocker volume switch with of the \ButtonVolDown{} button below the
  \ButtonVolUp{} button.  Above that is one more small round button, the \ButtonPower{}
  button, which is held down to turn the \dap{} on or off.}
  %
  \opt{mrobe100}{
  Hold the \dap{} with the black front facing you such that the m:robe writing 
  is readable. Below the writing is the touch sensitive pad with the 
  \ButtonMenu{}, \ButtonPlay{}, \ButtonLeft{}, \ButtonRight{} and \ButtonDisplay 
  controls indicated by their symbols. The dotted center strip is devided in 
  three parts: \ButtonUp{}, \ButtonSelect{} and \ButtonDown. On the top of the 
  unit, on the right, is the \ButtonPower{} switch, which is held down to turn 
  the \dap{} on or off.
  
  The \ButtonHold{} switch is located on the left of the \dap{}, below the 
  headphone socket. It puts the \dap{} into hold mode when it is switched to the 
  top of the unit. The buttons will have no effect when this is the case. On the 
  bottom of the unit, there is a connector for the docking station or the 
  proprietary USB connector for connecting directly to USB.}
  %
  \opt{iaudiom5,iaudiox5}{
  The \dap{} is curved so that the end with the screen on it is thicker than the 
  other end.  Hold the \dap{} wih the thick end towards the top and the screen
  facing towards you.  Half way up the front of the unit on the right hand side
  is a four way joystick which is the \ButtonUp{}, \ButtonDown{}, 
  \ButtonLeft{}, and \ButtonRight{} buttons. When pressed it serves as \ButtonSelect{}.
  
  On the right hand side of the \dap{} from top to bottom, first there is a two 
  way switch.  the \ButtonPower{} button is activated by pushing this switch up,
  and pushing this switch down until it clicks slightly will activate the 
  \ButtonHold{} button.  When the switch is in this position, none of the other
  keys will have an effect.
  
  Below the switch is a lozenge shaped button which is the \ButtonRec{} 
  button, and below that the final button on this side of the unit, the 
  \ButtonPlay{} button.  Just below this is a small hole which is difficult to
  locate by touch which is the internal microphone.  At the very bottom of 
  this side of the unit is the reset hole, which can be used to perform a hard
  reset by inserting a paper clip.
  
  On the bottom of the unit is the connector for the 
  \playerman{} subpack or dock.  On the top of the unit is a charge 
  indicator light, which may feel a bit like a button, but is not.
  
  From the top of the \dap{} on the left hand side is the headphone socket, then the 
  remote connector.  Below this is a cover which protects the \opt{iaudiox5}{USB
  host connector.}\opt{iaudiom5}{USB and charging connector}.}
  %
  \opt{e200,e200v2}{
  Hold the \dap{} with the turning wheel at the front and bottom.  On the bottom left
  of the front of the \dap{} is a raised round button, the \ButtonPower{} button.
  Above and to the left of this, on the outside of the turning wheel are four 
  buttons.  These are the \ButtonUp{}, \ButtonDown{}, \ButtonLeft{} and 
  \ButtonRight{} buttons.  Inside the wheel is the \ButtonSelect{} button.  Turning
  the wheel to the right activates the \ButtonScrollFwd{} function, and to the
  left, the \ButtonScrollBack{} function.  
  
  On the right of the unit is a slot for inserting flash cards.  On the bottom is 
  the connector for the USB cable.  On the left is the \ButtonRec{} button, and
  on the top, there is the headphone socket to the right, and the \ButtonHold{}
  switch.  Moving this switch to the right activates hold mode in which none of the
  other buttons have any effect.  Just to the left of the \ButtonHold{} switch is a
  small hole which contains the internal microphone.}
  %
  \opt{c200,c200v2}{
  Hold the \dap{} with the buttons on the right and the screen on the left. On
  the right side of the unit, there is a series of four connected buttons that
  form a square. The four sides of the square are the \ButtonUp{},
  \ButtonDown{}, \ButtonLeft{} and \ButtonRight{} buttons, respectively. Inside
  the square formed by these four buttons is the \ButtonSelect{} button. At the
  bottom right corner of the square is a small separate button, the
  \ButtonPower{} button.

  Moving clockwise around the outside of the unit, on the top are the \ButtonVolUp{}
  and \ButtonVolDown{} buttons, which control the volume of playback. The buttons can
  be distinguished by a sunken triangle on the \ButtonVolDown{} button, and a
  raised triangle on the \ButtonVolUp{} button. To the right of
  the volume buttons on the top of the unit is the slot for inserting flash
  memory cards. On the right side of the unit is the connector for the USB
  cable. At center of the bottom of the \dap{} is the \ButtonRec{} button. To
  the left of the \ButtonRec{} button is the \ButtonHold{} switch. Moving this
  switch to the right activates hold mode, in which none of the other buttons
  have any effect. On the lower left side of the unit is the headphone socket.
  Immediately above the headphone socket is a lanyard loop and the microphone.
  }
  %
  \opt{fuze,fuzev2}{
  Hold the \dap{} with the controls on the bottom and the screen on the top. The main
  controls are a scroll wheel with four clickable points and a button in the centre; pressing
  this centre button functions as \ButtonSelect{}. Going clockwise from the top, the clickable
  points on the wheel are the \ButtonUp{}, \ButtonRight{}, \ButtonDown{}, and \ButtonLeft{}
  buttons. Turning the wheel clockwise is \ButtonScrollFwd{}, and turning it counter-clockwise
  is \ButtonScrollBack{}. Immediately above and to the right of the wheel is the \ButtonHome{}
  button.

  On the lower left of the unit is a slot for inserting microSD cards. Immediately below that is
  the opening for the microphone.

  On the bottom of the unit is the connector for connecting a USB cable and the headphone socket.
  On the lower right hand side of the unit is a two-way switch. Pressing this switch up acts as
  \ButtonPower{}, and clicking it down until it locks acts as the \ButtonHold{} switch. When the
  \ButtonHold{} switch is on, none of the other buttons have any effect.
  }
  %
  \opt{clipplus,clipv1,clipv2}{
  Hold the \dap{} with the controls on the bottom and the screen on the top. The main
  controls are a four-way pad with a button in the centre; pressing this centre button
  functions as \ButtonSelect{}. Going clockwise from the top, the four-way pad contains
  the \ButtonUp{}, \ButtonRight{}, \ButtonDown{}, and \ButtonLeft{} buttons. 
  Immediately above and to the right of the four-way pad is the \ButtonHome{} button.
  }
  %
  \opt{clipplus}{
  The \ButtonPower{} button is on the top of the \dap{}, towards the right side.

  At the bottom of the right side of the \dap{} is a slot for microSD cards.
  Above this slot on the right side is the headphone socket.

  On the left hand panel is a two-way button that acts as \ButtonVolDown{} when
  pressed on the bottom, and \ButtonVolUp{} when pressed on the top. Immediately
  above the switch is a mini-USB port to connect the \dap{} to a computer.

  }
  %
  \opt{clipv1,clipv2}{
  On the left hand panel is a two way switch. Pressing this switch up acts as
  \ButtonPower{}, and clicking it down until it locks acts as the \ButtonHold{}
  switch. When the \ButtonHold{} switch is on, none of the other buttons have any
  effect. Immediately above the switch is a mini-USB port to connect the \dap{} to
  a computer.

  On the right hand panel is a two-way button that acts as \ButtonVolDown{} when
  pressed on the bottom, and \ButtonVolUp{} when pressed on the top. Immediately
  above this button is the headphone socket.
  }
  %
  \opt{vibe500}{
  Hold or lay the \dap{} so that the side with the controls and
  LCD is facing towards you. Below the LCD is the touch sensitive pad with the \ButtonMenu{}, 
  \ButtonPlay{}, \ButtonLeft{}, \ButtonRight{} controls and the scroll pad in the centre. The 
  scroll pad is oriented vertically between the \ButtonOK{} and \ButtonCancel{} buttons.
  Sliding a finger up or down the scroll pad acts as \ButtonUp{} and \ButtonDown{} respectively. 
  Note that the scroll pad is sensitive, so you will need to move 
  slowly at first and get a feel for how it works. 

  There are two buttons on the right hand side of the \dap{}: \ButtonPower{} on the top and 
  \ButtonRec{} underneath. Under these buttons, from top to bottom you can find: USB connector, 
  power connector and the reset hole if you need to perform a hardware reset. 

  The \ButtonHold{} switch is located on the left hand side of the \dap{}. Note that when the 
  \ButtonHold{} switch is moved towards the top of the \dap{}, hold is turned on and all the 
  other controls are disabled. Be sure \ButtonHold{} is off before trying to use your player. 

  On the top on the \dap{} is the internal microphone on the left and the line-in socket on the 
  right, near the headphone socket.}
  %
  \opt{player}{
  The main controls of this player are a four-way button on the right below
  the screen, and two round buttons to the left of it. Hold the \dap{} with
  these controls on the bottom and facing you.

  On the left hand side, the higher of the two small buttons is the \ButtonOn{},
  the lower of the two buttons is the \ButtonMenu{} button. The large circular
  button on the right contains, clockwise from the top, the \ButtonPlay{},
  the \ButtonRight{}, the \ButtonStop{}, and the \ButtonLeft{} buttons.

  On the top on the \dap{} is the headphone socket on the left and the line-out
  jack on the right. On the bottom of the \dap{} is the line-in jack on the left,
  the DC-In jack on the right, and the USB connector in the centre.
  }
  %
  \opt{recorder}{
  Holding the Jukebox in front of you, there should be three rectangular buttons
  in a horizontal line towards the middle of the unit, and below this to the left
  there is a circular four button array with the circular \ButtonPlay{} button
  as a fifth button in the centre. These are the navigation controls. Below the
  rectangular buttons and to the right of the circular buttons are two small round
  buttons one above the other.

  The \ButtonOn{} button is the topmost of the two buttons located below and to the
  left of the navigation controls whereas the lower of these two is called \ButtonOff.
  The small round button in the middle of the large circular button array is called
  \ButtonPlay{} button. To the right of the \ButtonPlay{} button there is the
  \ButtonRight{} button, left of it is the \ButtonLeft{}, above it \ButtonUp, and
  below the \ButtonPlay{} button there is the \ButtonDown{} button placed. In the row
  of three rectangular buttons the following buttons can be found (from left to right):
  \ButtonFOne{}, \ButtonFTwo{} and \ButtonFThree{}.

  On the top of the \dap{} is the headphone socket on the left and the line-out jack on
  the right. On the bottom of the \dap{} is the line-in jack on the left, the
  DC-In jack on the right, and the USB connector in the centre.
  }
  \opt{recorderv2fm}{
  Holding the Jukebox in front of you, there should be three rectangular buttons
  in a horizontal line towards the middle of the unit, and below this centred on the
  middle button there are four radial arc shaped buttons placed in a cross formation
  with the circular play button as the centre of the cross. These are the navigation
  controls. Below the cross and to the left are two other buttons.

  The \ButtonOn{} button is the leftmost of the two buttons located below and to the
  left of the navigation controls whereas the rightmost and little lower one of
  these two is called \ButtonOff{}. The round button raised slightly higher than the
  others in the centre of the navigation controls is the \ButtonPlay{} button.  To
  the right of the \ButtonPlay{} button  there is the \ButtonRight{} button, left of
  it is the \ButtonLeft{}, above it \ButtonUp{}, and below the \ButtonPlay{} button
  there is the \ButtonDown{} button  placed. In the row of three rectangular buttons
  the following buttons can be found (from left to right): \ButtonFOne{}, \ButtonFTwo{}
  and \ButtonFThree{}.
  }
}

\subsection{Turning the \dap{} on and off}
\opt{cowond2}{Rockbox has a dual-boot feature with the original firmware being
  the default.\\}
To turn on and off your Rockbox enabled \dap{} use the following keys:
    \begin{btnmap}
      \opt{IRIVER_H100_PAD,IRIVER_H300_PAD}{\ButtonOn}%
      \opt{MPIO_HD200_PAD}{Long \ButtonPlay}%
      \opt{IPOD_4G_PAD}{\ButtonMenu{} / \ButtonSelect}%
      \opt{IPOD_3G_PAD}{\ButtonMenu{} / \ButtonPlay}%
      \opt{ONDIO_PAD}{\ButtonOff}\opt{RECORDER_PAD,PLAYER_PAD}%
          {Long \ButtonOn}%
      \opt{IAUDIO_X5_PAD,IRIVER_H10_PAD,SANSA_E200_PAD,SANSA_C200_PAD,ONDA_VX777_PAD%
          ,GIGABEAT_PAD,MROBE100_PAD,GIGABEAT_S_PAD,sansaAMS,PBELL_VIBE500_PAD%
          }{\ButtonPower}%
      \opt{COWON_D2_PAD} {\ButtonPower{}, then \ButtonHold}%
      \opt{ONDA_VX777_PAD} {\ButtonPower{}}%
          &
      \opt{HAVEREMOTEKEYMAP}{
          \opt{IRIVER_RC_H100_PAD}{\ButtonRCOn}%
          \opt{IAUDIO_RC_PAD}{\ButtonRCPlay}
          &}
      Start Rockbox
          \\

      \opt{IRIVER_H100_PAD,IRIVER_H300_PAD}{Long \ButtonOff}%
      \opt{MPIO_HD200_PAD}{Long \ButtonPlay}%
      \opt{IPOD_4G_PAD,IPOD_3G_PAD}{Long \ButtonPlay}%
      \opt{ONDIO_PAD,recorderv2fm}{Long \ButtonOff}%
      \opt{recorder}{Double tap \ButtonOff\ when playback is stopped}%
      \opt{PLAYER_PAD}{From the Main Menu, select \textbf{Shutdown}}%
      \opt{IAUDIO_X5_PAD,IRIVER_H10_PAD,SANSA_E200_PAD,SANSA_C200_PAD%
          ,GIGABEAT_PAD,MROBE100_PAD,GIGABEAT_S_PAD,sansaAMS,COWON_D2_PAD%
          ,PBELL_VIBE500_PAD,ONDA_VX777_PAD}{Long \ButtonPower}%
          &
      \opt{HAVEREMOTEKEYMAP}{ 
          \opt{IRIVER_RC_H100_PAD}{Long \ButtonRCStop}%
          \opt{IAUDIO_RC_PAD}{Long \ButtonRCPlay}
          &}
      Shutdown Rockbox
          \\
    \end{btnmap}

\label{ref:Safeshutdown}On shutdown, Rockbox automatically saves its settings.

\opt{IRIVER_H100_PAD,IRIVER_H300_PAD,IAUDIO_X5_PAD,SANSA_E200_PAD%
  ,SANSA_C200_PAD,IRIVER_H10_PAD,IPOD_4G_PAD,GIGABEAT_PAD}{%
  If you have problems with your settings, such as accidentally having
  set the colours to black on black, they can be reset at boot time.  See
  the Reset Settings in \reference{ref:manage_settings_menu} for details.
}%

\opt{PLAYER_PAD,RECORDER_PAD,ONDIO_PAD,GIGABEAT_PAD,IPOD_4G_PAD,SANSA_E200_PAD%
,SANSA_C200_PAD,IAUDIO_X5_PAD,IAUDIO_M5_PAD,IPOD_3G_PAD}{%
  In the unlikely event of a software failure, hardware poweroff or reset can be
  performed by holding down \opt{PLAYER_PAD}{\ButtonStop}\opt{RECORDER_PAD,ONDIO_PAD}
  {\ButtonOff}\opt{GIGABEAT_PAD}{the battery switch}\opt{IPOD_4G_PAD}
  {\ButtonMenu{} and \ButtonSelect{} simultaneously}%
  \opt{IPOD_3G_PAD}{\ButtonMenu{} and \ButtonPlay{} simultaneously}%
  \opt{SANSA_E200_PAD,SANSA_C200_PAD,IAUDIO_X5_PAD,IAUDIO_M5_PAD}
  {\ButtonPower} until the \dap{} shuts off or reboots.
}%
\opt{IRIVER_H100_PAD,IRIVER_H300_PAD,IAUDIO_M3_PAD,IRIVER_H10_PAD,MROBE100_PAD
,PBELL_VIBE500_PAD,MPIO_HD200_PAD}{%
  In the unlikely event of a software failure, a hardware reset can be
  performed by inserting a paperclip gently into the Reset hole.
}%

\nopt{gigabeatf,iaudiom3,iaudiom5,iaudiox5,archos}
  {
  \subsection{Starting the original firmware}
  \label{ref:Dualboot}
  \opt{ipod4g,ipodcolor,ipodvideo,ipodnano,ipodnano2g,ipodmini}
    {
    Rockbox has a dual-boot feature. To boot into the original firmware, shut
    down the device as described above. Turn on the \ButtonHold{} switch
    immediately after turning the player on. The Apple logo will
    display for a few seconds as Rockbox loads the original firmware.
    
    You can also load the original firmware by shutting down the device,
    then clicking the \ButtonHold{} switch on and connecting the iPod
    to your computer.
 
    Regardless of which method you use to boot to the original firmware, you can
    return to Rockbox by pressing and holding \ButtonMenu{} and \ButtonSelect{}
    simultaneously until the player hard resets.
    }

  \opt{ipod1g2g,ipod3g}
    {
    Rockbox has a dual-boot feature. To boot into the original firmware, shut
    down the device as described above. Turn on the \ButtonHold{} switch
    immediately after turning the player on. The Apple logo will
    display for a few seconds as Rockbox loads the original firmware.
    
    You can also load the original firmware by shutting down the device,
    then clicking the \ButtonHold{} switch on and connecting the iPod
    to your computer.
 
    Regardless of which method you use to boot to the original firmware, you can
    return to Rockbox by pressing and holding \ButtonMenu{} and \ButtonPlay{}
    simultaneously until the player hard resets.
    }

  \opt{iriverh100,iriverh300}
    {
    Rockbox has a dual-boot feature. To boot into the original firmware,
    when the \dap{} is turned off, press and hold the \ButtonRec{} button,
    and then press the \ButtonOn{} button.
    }

  \opt{mpiohd200}
    {
    Rockbox has a dual-boot feature. To boot into the original firmware,
    when the \dap{} is turned off, press and hold the \ButtonRec{} button,
    and then press the \ButtonPlay{} button. This will bring you to the
    short menu where you can choose among: Boot Rockbox, Boot MPIO firmware
    and Shutdown. Select the option you need with \ButtonRew{} and \ButtonFF{}
    and confirm with long \ButtonPlay{}.
    }
  \opt{iriverh10,iriverh10_5gb}
    {
    Rockbox has a dual-boot feature. It loads the original firmware from
    the file \fname{/System/OF.mi4}. To boot into the original firmware,
    press and hold the \ButtonLeft{} button while turning on the player.
    \note{The iriver firmware does not shut down properly when you turn it off,
    it only goes to sleep. To get back into Rockbox when exiting from the
    iriver firmware, you will need to reset the player by \opt{iriverh10}{%
    inserting a pin in the reset hole}\opt{iriverh10_5gb}{removing and
    reinserting the battery}.}
    }
    
  \opt{sansa,sansaAMS}
    {
    Rockbox has a dual-boot feature. To boot into the original firmware,
    press and hold the \ButtonLeft{} button while turning on the player.
    }

  \opt{clipv2,fuzev2,clipplus}
    {
    The player will always boot into the original firmware if it is powered
    by a USB connection.  When Rockbox USB support is officially stable,
    a new bootloader will be released changing this behavior.
    }

  \opt{mrobe100}
    {
    Rockbox has a dual-boot feature. It loads the original firmware from
    the file \fname{/System/OF.mi4}. To boot into the original firmware,
    when the \dap{} is turned off, press the \ButtonPower{} button once and then 
    a second time when the m:robe bootlogo (the headphone) appears. Hold the
    \ButtonPower{} button until you see the ``Loading original firmware...'' 
    message on the screen.
    }

  \opt{gigabeats}
    {
    Rockbox has a dual-boot feature. To boot into the original firmware,
    turn the \ButtonHold{} switch on just after turning on the \dap{}.
    To return to Rockbox, shutdown the \dap{}, then turn the battery switch
    on the bottom off then on again. Rockbox should now start.
    }

  \opt{cowond2}
    {
    Use \ButtonPower{} to boot the original \playerman{} firmware.
    }

  \opt{vibe500}
    {
    Rockbox has a dual-boot feature where it is possible to load the original firmware from
    the file \fname{/System/OF.mi4}. To boot into the original firmware press and release
    \ButtonPower{} and then immediately after the backlight turns on, press the \ButtonOK{}
    button and keep it pressed until the original firmware starts.
    }

  \opt{ondavx777}
    {
    Rockbox has a dual-boot feature where it is possible to load the original firmware from
    the file \fname{/SD/ccpmp.bin}. To boot into the original firmware press and release
    \ButtonPower{} immediately after the Rockbox Logo appear on the screen.
    }

  }
\subsection{Putting music on your \dap{}}

\opt{usb_hid}{
\note{Due to a bug in some OS X versions, the \dap{} can not be mounted, unless
    the USB HID feature is disabled. See \reference{ref:USB_HID} for more
    information.\newline
}
}

With the \dap{} connected to the computer as an MSC/UMS device (like a
USB Drive), music files can be put on the player via any standard file
transfer method that you would use to copy files between drives (e.g. Drag-and-Drop).
Files may be placed wherever you like on the \dap{}, but it is strongly
suggested \emph{NOT} to put them in the \fname{/.rockbox} folder and instead 
put them in any other folder, e.g. \fname{/}, \fname{/music} or \fname{/audio}.
The default directory structure that is assumed by some parts of Rockbox
\opt{albumart}{%
    (album art searching, and missing-tag fallback in some WPSes) uses the
    parent directory of a song as the Album name, and the parent directory of
    that folder as the Artist name. WPSes may display information incorrectly if
    your files are not properly tagged, and you have your music organized in a
    way different than they assume when attempting to guess the Artist and Album
    names from your filetree. See \reference{ref:album_art} for the requirements
    for Album Art to work properly. 
}%
\nopt{albumart}{%
    (missing-tag fallback in some WPSes) uses the parent directory of a song
    as the Album name, and the parent directory of that folder as the Artist
    name. WPSes may display
    information incorrectly if your files are not properly tagged, and you have
    your music organized in a way different than they assume when attempting to
    guess the Artist and Album names from your filetree.
}%
\opt{swcodec}{
    See \reference{ref:Supportedaudioformats} for a list of supported audio
    formats.
}

\subsection{The first contact}

After you have first started the \dap{}, you'll be presented by the
\setting{Main Menu}. From this menu you can reach every function of Rockbox,
for more information (see \reference{ref:main_menu}). To browse the files
on your \dap{}, select \setting{Files} (see \reference{ref:file_browser}), and to
browse in a view that is based on the meta-data\footnote{ID3 Tags, Vorbis
comments, etc.} of your audio files, select \setting{Database} (see
\reference{ref:database}).

\subsection{Basic controls}
When browsing files and moving through menus you usually get a list view
presented. The navigation in these lists are usually the same and should be
pretty intuitive.
In the tree view use \ActionStdNext{} and \ActionStdPrev{} to move around
the selection. Use \ActionStdOk{} to select an item. \opt{wheel_acceleration}{
Note that the scroll speed is accelerating the faster you rotate the wheel.}
When browsing the file system selecting an audio file plays it. The view 
switches to the ``While playing screen'', usually abbreviated as ``WPS'' (see 
\reference{ref:WPS}. The dynamic playlist gets replaced with the contents of 
the current directory. This way you can easily treat directories as playlists. 
The created dynamic playlist can be extended or modified while playing. This is 
also known as ``on-the-fly playlist''.
To go back to the \setting{File Browser} stop the playback with the
\ActionWpsStop{} button or return to the file browser while keeping playback
running using \ActionWpsBrowse{}.
In list views you can go back one step with \ActionTreeParentDirectory.

\subsection{Basic concepts}
\subsubsection{Playlists}
Rockbox is playlist oriented. This means that every time you play an audio file,
a so-called ``dynamic playlist'' is generated, unless you play a saved
playlist. You can modify the dynamic playlist while playing and also save
it to a file. If you do not want to use playlists you can simply play your
files directory based.
Playlists are covered in detail in \reference{ref:working_with_playlists}.

\subsubsection{Menu}
From the menu you can customise Rockbox. Rockbox itself is very customisable.
Also there are some special menus for quick access to frequently used
functions.

\subsubsection{Context Menu}
Some views, especially the file browser and the WPS have a context menu.
From the file browser this can be accessed with \ActionStdContext{}.
The contents of the context menu vary, depending on the situation it gets
called. The context menu itself presents you with some operations you can
perform with the currently highlighted file. In the file browser this is
the file (or directory) that is highlighted by the cursor. From the WPS this is
the currently playing file. Also there are some actions that do not apply
to the current file but refer to the screen from which the context menu
gets called. One example is the playback menu, which can be called using
the context menu from within the WPS.

\section{Customising Rockbox}
Rockbox' User Interface can be customised using ``Themes''. Themes usually
only affect the visual appearance, but an advanced user can create a theme
that also changes various other settings like file view, LCD settings and
all other settings that can be modified using \fname{.cfg} files. This topic
is discussed in more detail in \reference{ref:manage_settings}.
The Rockbox distribution comes with some themes that should look nice on
your \dap{}.

\opt{lcd_bitmap}{
\note{Some of the themes shipped with Rockbox need additional
fonts from the fonts package, so make sure you installed them.
Also, if you downloaded additional themes from the Internet make sure you
have the needed fonts installed as otherwise the theme may not display
properly.}
}

\nopt{ondio}{
  \opt{usb_power}{
    \section{USB Charging}
    \nopt{clipv2,fuzev2,clipplus}{
    To charge your \dap{} over USB, hold any button while plugging it
    in. This will prevent it from connecting to your computer and let you
    continue to use it normally. Your \dap{} must already be in Rockbox for this
    to function.
    \note{Be aware that this button may still perform its normal function, so
    it is recommended to use a button without harmful side effects, such as
    \ActionStdUsbCharge{}.}
    
    }
    \opt{clipv2,fuzev2,clipplus}{
    Your \dap{} will automatically charge over USB if the cable is plugged in
    while Rockbox is running.
    } 
  }
}

\opt{ondio}{
  \section{USB Power}

    To power your \dap{} over USB, hold \ActionStdUsbCharge{} while plugging it
    in. This will prevent it from connecting to your computer and let you
    continue to use it normally. Your \dap{} must already be in Rockbox for this
    to function.
}

% $Id$ %
\chapter{Browsing and playing}
\section{\label{ref:file_browser}File Browser}
\screenshot{rockbox_interface/images/ss-file-browser}{The file browser}{}
Rockbox lets you browse your music in either of two ways. The 
\setting{File Browser} lets you navigate through the files and directories on 
your \dap, entering directories and executing the default action on each file.
To help differentiate files, each file format is displayed with an icon. 

The \setting{Database Browser}, on the other hand, allows you to navigate 
through the music on your player using categories like album, artist, genre,
etc.

You can select whether to browse using the \setting{File Browser} or the
\setting{Database Browser} by selecting either \setting{Files} or
\setting{Database} in the \setting{Main Menu}.
If you choose the \setting{File Browser}, the \setting{Show Files} setting
lets you select what types of files you wish to view. See
\reference{ref:ShowFiles} for more information on the \setting{Show Files}
setting.

\note{The \setting{File Browser} allows you to manipulate your files in ways
that are not available within the \setting{Database Browser}. Read more about
\setting{Database} in \reference{ref:database}. The remainder of this section
deals with the \setting{File Browser}.}

\opt{ondio}{
Unlike the Archos Firmware, Rockbox provides multivolume support for the
MultiMediaCard, this means the \dap{} can access both data volumes (internal
memory and the MMC), thus being able to for instance, build playlists with
files from both volumes.
In the \setting{File Browser} a new directory will appear as soon as the device
has read the content after inserting the card. This new directory's name is
generated as \fname{<MMC1>}, and will behave exactly as any other directory
on the \dap{}.
}

\opt{iriverh10,iriverh10_5gb}{\note{
If your \dap{} is a MTP model, the Music directory where all your music is stored
may be hidden in the \setting{File Browser}. This may be fixed by either
changing its properties (on a computer) to not hidden, or by changing
the \setting{Show Files} setting to all.
}}

\subsection{\label{ref:controls}File Browser Controls}
\begin{btnmap}
      \ActionStdPrev{}/\ActionStdNext{}
      \opt{HAVEREMOTEKEYMAP}{& \ActionRCStdPrev{}/\ActionRCStdNext{}}
         & Go to previous/next item in list. If you are on the first/last 
           entry, the cursor will wrap to the last/first entry.\\
      %
      \opt{IRIVER_H100_PAD,IRIVER_H300_PAD,RECORDER_PAD}
        {
          \ButtonOn+\ButtonUp{}/ \ButtonDown
          \opt{HAVEREMOTEKEYMAP}{&
            \opt{IRIVER_RC_H100_PAD}{\ButtonRCSource{}/ \ButtonRCBitrate}
          }
          & Move one page up/down in the list.\\
        }
      \opt{IRIVER_H10_PAD}
        {
          \ButtonRew{}/ \ButtonFF
          & Move one page up/down in the list.\\
        }
      %
      \ActionTreeParentDirectory
      \opt{HAVEREMOTEKEYMAP}{& \ActionRCTreeParentDirectory}
      & Go to the parent directory.\\
      %
      \ActionTreeEnter
      \opt{HAVEREMOTEKEYMAP}{& \ActionRCTreeEnter}
      & Execute the default action on the selected file or enter a
        directory.\\
      %
      \ActionTreeWps 
      \opt{HAVEREMOTEKEYMAP}{& \ActionRCTreeWps}
         & If there is an audio file playing, return to the
         \setting{While Playing Screen} (WPS) without stopping playback.\\
      %
      \nopt{player,SANSA_C200_PAD}%
        {%
          \ActionTreeStop 
          \opt{HAVEREMOTEKEYMAP}{& \ActionRCTreeStop}
          & Stop audio playback.\\%
        }%
      %
      \ActionStdContext{}
      \opt{HAVEREMOTEKEYMAP}{& \ActionRCStdContext}
      & Enter the \setting{Context Menu}.\\
      %
      \ActionStdMenu{}
      \opt{HAVEREMOTEKEYMAP}{& \ActionRCStdMenu}
      & Enter the \setting{Main Menu}.\\
      %
      \opt{quickscreen}{
        \ActionStdQuickScreen
        \opt{HAVEREMOTEKEYMAP}{& \ActionRCStdQuickScreen}
        & Switch to the \setting{Quick Screen}
        (see \reference{ref:QuickScreen}). \\
      }
      \opt{RECORDER_PAD}{
        \ButtonFThree & Switch to the \setting{Quick Screen}.\\ 
        %
      }
      %
      \opt{SANSA_E200_PAD}{
        \ActionStdRec & Switch to the \setting{Recording Screen}.\\
      %
      }
      \nopt{touchscreen}{\opt{hotkey}{
        \ActionTreeHotkey
            &
        \opt{HAVEREMOTEKEYMAP}{
            &}
        Activate the \setting{Hotkey} function
        (see \reference{ref:Hotkeys}).
            \\
      }}
\end{btnmap}

\opt{RECORDER_PAD}{
  The functions of the F keys are also summarised on the button bar at the
  bottom of the screen.
}

\subsection{\label{ref:Contextmenu}\label{ref:PartIISectionFM}Context Menu}
\screenshot{rockbox_interface/images/ss-context-menu}{The Context Menu}{}

The \setting{Context Menu} allows you to perform certain operations on files or 
directories.  To access the \setting{Context Menu}, position the selector over a file 
or directory and access the context menu with \ActionStdContext{}.\\

\note{The \setting{Context Menu} is a context sensitive menu.  If the 
\setting{Context Menu} is invoked on a file, it will display options available 
for files.  If the \setting{Context Menu} is invoked on a directory, 
it will display options for directories.\\}

The \setting{Context Menu} contains the following options (unless otherwise noted, 
each option pertains both to files and directories):

\begin{description}
\item [Playlist.]
  Enters the \setting{Playlist Submenu} (see \reference{ref:playlist_submenu}).
\item [Playlist Catalogue.]
  Enters the \setting{Playlist Catalogue Submenu} (see 
  \reference{ref:playlist_catalogue}).
\item [Rename.]
  This function lets the user modify the name of a file or directory.
\item [Cut.]
  Copies the name of the currently selected file or directory to the clipboard
  and marks it to be `cut'.
\item [Copy.]
  Copies the name of the currently selected file or directory to the clipboard
  and marks it to be `copied'.
\item [Paste.]
  Only visible if a file or directory name is on the clipboard. When selected
  it will move or copy the clipboard to the current directory.
\item [Delete.]
  Deletes the currently selected file. This option applies only to files, and
  not to directories. Rockbox will ask for confirmation before deleting a file.
  Press \ActionYesNoAccept{}
  to confirm deletion or any other key to cancel.
\item [Delete Directory.]
  Deletes the currently selected directory and all of the files and subdirectories
  it may contain. Deleted directories cannot be recovered. Use this feature with
  caution!
\opt{lcd_non-mono}{
\item [Set As Backdrop.]
  Set the selected \fname{bmp} file as background image. The bitmaps need to meet the
  conditions explained in \reference{ref:LoadingBackdrops}.
}
\item [Open with.]
  Runs a viewer plugin on the file. Normally, when a file is selected in Rockbox,
  Rockbox automatically detects the file type and runs the appropriate plugin.
  The \setting{Open With} function can be used to override the default action and
  select a viewer by hand.
  For example, this function can be used to view a text file
  even if the file has a non-standard extension (i.e., the file has an extension
  of something other than \fname{.txt}). See \reference{ref:Viewersplugins}
  for more details on viewers.
\item [Create Directory.]
  Create a new directory in the current directory on the disk.
\item [Properties.]
  Shows properties such as size and the time and date of the last modification
  for the selected file. If used on a directory, the number of files and
  subdirectories will be shown, as well as the total size.
\opt{recording}{
  \item [Set As Recording Directory.]
    Save recordings in the selected directory.
}
\item [Add to Shortcuts.]
  Adds a link to the selected item in the \fname{shortcuts.link} file.
  If the file does not already exist it will be created in the root directory.
  Note that if you create a shortcut to a file, Rockbox will not open it upon
  selecting, but simply bring you to its location in the \setting{File Browser}.
\end{description}

\subsection{\label{sec:virtual_keyboard}Virtual Keyboard}
\screenshot{rockbox_interface/images/ss-virtual-keyboard}{The virtual keyboard}{}
This is the virtual keyboard that is used when entering text in Rockbox, for 
example when renaming a file or creating a new directory.
\nopt{player}{The virtual keyboard can be easily changed by making a text file
 with the required layout. More information on how to achieve this can be found
 on the Rockbox website at \wikilink{LoadableKeyboardLayouts}.}

\opt{morse_input}{
  Also you can switch to Morse code input mode by changing the
  \setting{Use Morse Code Input} setting%
  \opt{IRIVER_H100_PAD,IRIVER_H300_PAD,IPOD_4G_PAD,IPOD_3G_PAD,IRIVER_H10_PAD%
      ,GIGABEAT_PAD,GIGABEAT_S_PAD,MROBE100_PAD,SANSA_E200_PAD,PBELL_VIBE500_PAD}%
    { or by pressing \ActionKbdMorseInput{} in the virtual keyboard}%
  .}

\nopt{player}{% no "Actions" yet in the Player's virtual keyboard

\note{When the cursor is on the input line, \ActionKbdSelect{} deletes the preceding character}

\begin{btnmap}
    \opt{IRIVER_H100_PAD,IRIVER_H300_PAD,RECORDER_PAD,GIGABEAT_PAD,GIGABEAT_S_PAD%
        ,MROBE100_PAD,SANSA_E200_PAD,SANSA_FUZE_PAD,SANSA_C200_PAD}{
        \ActionKbdCursorLeft{} / \ActionKbdCursorRight
            &
        \opt{HAVEREMOTEKEYMAP}{\ActionRCKbdCursorLeft{} / \ActionRCKbdCursorRight
            &}
        Move the line cursor within the text line.
            \\
        %
        \ActionKbdBackSpace
            &
        \opt{HAVEREMOTEKEYMAP}{
            &}
        Delete the character before the line cursor.
            \\
    }%
    \ActionKbdLeft{} / \ActionKbdRight
        &
    \opt{HAVEREMOTEKEYMAP}{\ActionRCKbdLeft{} / \ActionRCKbdRight
        &}
    Move the cursor on the virtual keyboard.
    If you move out of the picker area, you get the previous/next page of
    characters (if there is more than one).
        \\
    %
    \ActionKbdUp{} / \ActionKbdDown
        &
    \opt{HAVEREMOTEKEYMAP}{\ActionRCKbdUp{} / \ActionRCKbdDown
        &}
    Move the cursor on the virtual keyboard.
    If you move out of the picker area you get to the line edit mode.
        \\
    %
    \nopt{IPOD_3G_PAD,IPOD_4G_PAD,IRIVER_H10_PAD,ONDIO_PAD,PBELL_VIBE500_PAD}{
        \ActionKbdPageFlip
            &
        \opt{HAVEREMOTEKEYMAP}{\ActionRCKbdPageFlip
            &}
        Flip to the next page of characters (if there is more than one).
            \\
    }
    %
    \ActionKbdSelect
        &
    \opt{HAVEREMOTEKEYMAP}{\ActionRCKbdSelect
        &}
    Insert the selected keyboard letter at the current line cursor position.
        \\
    %
    \ActionKbdDone
        &
    \opt{HAVEREMOTEKEYMAP}{\ActionRCKbdDone
        &}
    Exit the virtual keyboard and save any changes.
        \\
    %
    \ActionKbdAbort
        &
    \opt{HAVEREMOTEKEYMAP}{\ActionRCKbdAbort
        &}
    Exit the virtual keyboard without saving any changes.
        \\
% to be done - create a separate section for morse imput and update the info
      \opt{morse_input}{
        \opt{IRIVER_H100_PAD,IRIVER_H300_PAD,GIGABEAT_PAD,GIGABEAT_S_PAD,MROBE100_PADD%
            ,SANSA_E200_PA,IPOD_4G_PAD,IPOD_3G_PAD,IRIVER_H10_PAD,PBELL_VIBE500_PAD}{
          \ActionKbdMorseInput
          \opt{HAVEREMOTEKEYMAP}{& \ActionRCKbdMorseInput}
          & Toggle keyboard input mode and Morse code input mode. \\}
        %
        \ActionKbdMorseSelect
        \opt{HAVEREMOTEKEYMAP}{& \ActionRCKbdMorseSelect}
        & Tap to select a character in Morse code input mode. \\
      } 
\end{btnmap}
}% end of non-Player section

\opt{player}{
  The current text line to be entered or edited is always listed on the first
  line of the display. The second line of the display can contain the character
  selection bar, as in the screenshot above.
    \begin{btnmap}
      \ButtonOn & Toggle picker- and line edit mode. \\
      \ButtonLeft{} / \ButtonRight
        & Move back and forth in the selected line (picker of input line). \\
      \ButtonPlay
        & Pick character in character bar, or act as backspace in the text line. \\
      Long \ButtonPlay & Accept \\
      \ButtonStop & Cancel \\
      \ButtonMenu & Flip picker lines. \\
    \end{btnmap}
}

\input{rockbox_interface/tagcache.tex}
% $Id$ %
\section{\label{ref:WPS}While Playing Screen}
The While Playing Screen (WPS) displays various pieces of information about the
currently playing audio file.
%
\opt{lcd_bitmap}{%
  The appearance of the WPS can be configured using WPS configuration files.
  The items shown depend on your configuration -- all items can be turned on
  or off independently. Refer to \reference{ref:wps_tags} for details on how
  to change the display of the WPS.
  \begin{itemize}
    \nopt{ondio}{
    \item Status bar: The Status bar shows Battery level, charger status, 
      volume, play mode, repeat mode, shuffle mode\opt{rtc}{ and clock}.
      In contrast to all other items, the status bar is always at the top of
      the screen.
    }
    \opt{ondio}{
    \item Status bar: The Status bar shows Battery level, USB power mode, key
      lock status, memory access indicator. In contrast to all other items, the
      status bar is always at the top of the screen.
    }
  \item (Scrolling) path and filename of the current song.
  \item The ID3 track name.
  \item The ID3 album name.
  \item The ID3 artist name.
  \item Bit rate. VBR files display average bitrate and ``(avg)''
  \item Elapsed and total time.
  \item A slidebar progress meter representing where in the song you are.
  \item Peak meter.
  \end{itemize}
}
\opt{recorder,recorderv2fm,ondio}{
  \note{
  \begin{itemize}
  \item The number of lines shown depends on the size of the font used.
  \item The peak meter is only visible if you turn off the status bar or if
    using a small font that gives 8 or more display lines.
  \end{itemize}
  }
}
%
\opt{player}{
  \note{
  \begin{itemize}
  \item Playlist index/Playlist size: Artist {}- Title.
  \item Current{}-time Progress{}-indicator Left.
  \end{itemize}
  }
}

See \reference{ref:ConfiguringtheWPS} for details of customising
your WPS (While Playing Screen).


\subsection{\label{ref:WPS_Key_Controls}WPS Key Controls}

  \begin{btnmap}
      \ActionWpsVolUp{} / \ActionWpsVolDown
      \opt{HAVEREMOTEKEYMAP}{& \ActionRCWpsVolUp{} / \ActionRCWpsVolDown}
      & Volume up/down.\\
      %
      \ActionWpsSkipPrev
       \opt{HAVEREMOTEKEYMAP}{& \ActionRCWpsSkipPrev}
      & Go to beginning of track, or if pressed while in the
        first seconds of a track, go to the previous track.\\
      %
      \ActionWpsSeekBack
      \opt{HAVEREMOTEKEYMAP}{& \ActionRCWpsSeekBack}
      & Rewind in track.\\
      %
      \ActionWpsSkipNext
      \opt{HAVEREMOTEKEYMAP}{& \ActionRCWpsSkipNext}
      & Go to the next track.\\
      %
      \ActionWpsSeekFwd
      \opt{HAVEREMOTEKEYMAP}{& \ActionRCWpsSeekFwd}
      & Fast forward in track.\\
      %
      \ActionWpsPlay
      \opt{HAVEREMOTEKEYMAP}{& \ActionRCWpsPlay}
      & Toggle play/pause.\\
      %
      \ActionWpsStop 
      \opt{HAVEREMOTEKEYMAP}{& \ActionRCWpsStop}
      & Stop playback.\\
      %
      \ActionWpsBrowse
      \opt{HAVEREMOTEKEYMAP}{& \ActionRCWpsBrowse}
      & Return to the \setting{File Browser} / \setting{Database}.\\
      %
      \ActionWpsContext
      \opt{HAVEREMOTEKEYMAP}{& \ActionRCWpsContext}
      & Enter \setting{WPS Context Menu}.\\
      %
      \opt{ONDIO_PAD}{\ActionWpsContext{} twice}%
      \nopt{ONDIO_PAD}{\ActionWpsMenu}%
      \opt{HAVEREMOTEKEYMAP}{& \ActionRCWpsMenu}
      & Enter \setting{Main Menu}%
      \opt{ONDIO_PAD}{ via the \setting{WPS Context Menu}}.\\%
      %
      \opt{quickscreen}{%
        \ActionWpsQuickScreen
        \opt{HAVEREMOTEKEYMAP}{& \ActionRCWpsQuickScreen}
          & Switch to the \setting{Quick Screen}
          (see \reference{ref:QuickScreen}). \\}%
      %
      % software hold targets
      \nopt{hold_button}{%
          \opt{RECORDER_PAD}{\ButtonFOne+\ButtonDown}
          \opt{PLAYER_PAD}{\ButtonMenu+\ButtonStop}
          \opt{ONDIO_PAD}{\ButtonMenu+\ButtonDown}
          \opt{SANSA_CLIP_PAD}{\ButtonHome+\ButtonSelect}
          & Key lock (software hold switch) on/off.\\
      }%
      %These actions need definitions for the other targets
      \opt{RECORDER_PAD}{%
        \ButtonFThree & Toggles Display quick screen.\\%
        \ButtonFOne+\ButtonPlay & Mute on/off.\\%
      }%
      \opt{PLAYER_PAD}{%
        \ButtonMenu+\ButtonPlay & Mute on/off.\\%
      }%
      % We explicitly list all the appropriate targets here and do no condition
      % on the 'pitchscreen' feature since some players have the feature but do
      % not have the button to go from the WPS to the pitch screen.
      \opt{RECORDER_PAD,IRIVER_H100_PAD,IRIVER_H300_PAD,IRIVER_H10_PAD,MROBE100_PAD%
	           ,GIGABEAT_PAD,GIGABEAT_S_PAD,SANSA_E200_PAD,SANSA_C200_PAD}{%
        \ActionWpsPitchScreen
        \opt{HAVEREMOTEKEYMAP}{& \ActionRCWpsPitchScreen}
          & Show \setting{Pitch Screen} (see \reference{sec:pitchscreen}).\\%
      }%
      \opt{GIGABEAT_PAD,GIGABEAT_S_PAD,SANSA_CLIP_PAD,MROBE100_PAD,PBELL_VIBE500_PAD}{%
        \ActionWpsPlaylist
        \opt{HAVEREMOTEKEYMAP}{&}
          & Show current \setting{Playlist}.\\%
      }%
      \opt{RECORDER_PAD,IRIVER_H100_PAD,IRIVER_H300_PAD,IRIVER_H10_PAD%
          ,SANSA_E200_PAD,SANSA_C200_PAD}{%
        \ActionWpsIdThreeScreen 
          \opt{HAVEREMOTEKEYMAP}{& \ActionRCWpsIdThreeScreen}
          & Enter \setting{ID3 Viewer}.\\%
      }%
      \opt{hotkey}{%
        \ActionWpsHotkey \opt{HAVEREMOTEKEYMAP}{& }
        & Activate the \setting{Hotkey} function (see \reference{ref:Hotkeys}).\\
      }
      \opt{ab_repeat_buttons}{%
         \ActionWpsAbSetBNextDir{} or }%
         % not all targets have the above action defined but the one below works on all
      Short \ActionWpsSkipNext{} + Long \ActionWpsSkipNext
      \opt{HAVEREMOTEKEYMAP}{
        &
          \opt{IRIVER_RC_H100_PAD}{\ActionRCWpsAbSetBNextDir{} or}
        Short \ActionRCWpsSkipNext{} + Long \ActionRCWpsSkipNext}
      & Skip to the next directory.\\
      %
      \opt{ab_repeat_buttons}{%
         \ActionWpsAbSetAPrevDir{} or }%
      Short \ActionWpsSkipPrev{} + Long \ActionWpsSkipPrev
      \opt{HAVEREMOTEKEYMAP}{
        &
          \opt{IRIVER_RC_H100_PAD}{\ActionRCWpsAbSetAPrevDir{} or}
        Short \ActionRCWpsSkipPrev{} + Long \ActionRCWpsSkipPrev}
      & Skip to the previous directory.\\
      %
      \opt{SANSA_E200_PAD,SANSA_C200_PAD,IRIVER_H100_PAD,IRIVER_H300_PAD}{
        \ActionStdRec
          \opt{HAVEREMOTEKEYMAP}{&} 
          & Switch to the \setting{Recording Screen}.\\
      }%
  \end{btnmap}


\opt{lcd_bitmap}{
  \subsection{\label{ref:peak_meter}Peak Meter}
  The peak meter can be displayed on the While Playing Screen and consists of
  several indicators. 
  \opt{recording}{
    For a picture of the peak meter, please see the While
    Recording Screen in \reference{ref:while_recording_screen}.
  }
  
  \begin{description}
  \item [The bar:]
    This is the wide horizontal bar. It represents the current volume value.
  \item [The peak indicator:]
    This is a little vertical line at the right end of the bar. It indicates 
    the peak volume value that occurred recently.
  \item [The clip indicator:]
    This is a little black block that is displayed at the very right of the
    scale when an overflow occurs. It usually does not show up during normal
    playback unless you play an audio file that is distorted heavily.
    \opt{recording}{
      If you encounter clipping while recording, your recording will sound distorted.
      You should lower the gain.
    }
    \note{Note that the clip detection is not very precise.
     Clipping might occur without being indicated.}
  \item [The scale:]
    Between the indicators of the right and left channel there are little dots.
    These dots represent important volume values. In linear mode each dot is a
    10\% mark. In dBFS mode the dots represent the following values (from right
    to left): 0~dB, {}-3~dB, {}-6~dB, {}-9~dB, {}-12~dB, {}-18~dB, {}-24~dB, {}-30~dB,
    {}-40~dB, {}-50~dB, {}-60~dB.
  \end{description}
}
\subsection{\label{sec:contextmenu}The WPS Context Menu}
Like the context menu for the \setting{File Browser}, the \setting{WPS Context Menu} 
allows you quick access to some often used functions.

\subsubsection{Playlist}
The \setting{Playlist} submenu allows you to view, save, search and
reshuffle the current playlist. To change settings for the
\setting{Playlist Viewer} press \ActionStdContext{} while viewing the current
playlist to bring up the \setting{Playlist Viewer Menu}. In this menu, you
can find the \setting{Playlist Viewer Settings}.

\subsubsection{Playlist Viewer Settings}
  \begin{description}
    \item[Show Icons.] This toggles display of the icon for the currently 
    selected playlist entry and the icon for moving a playlist entry
    \item[Show Indices.] This toggles display of the line numbering for
       the playlist
    \item[Track Display.] This toggles between filename only and full path
       for playlist entries
    \item[Save Current Playlist.] Allows the current playlist to be saved as
       a \fname{.m3u8} playlist file
  \end{description}

    
\subsubsection{Playlist catalogue}
  \begin{description}
    \item [View catalogue.] This lists all playlists that are part of the
    Playlist catalogue. You can load a new playlist directly from this list.
    \item [Add to playlist.] Adds the currently playing file to a playlist.
    Select the playlist you want the file to be added to and it will get
    appended to that playlist.
    \item [Add to new playlist.] Similar to the previous entry this will
    add the currently playing track to a playlist. You need to enter a name
    for the new playlist first.
  \end{description}

\subsubsection{Sound Settings}
This is a shortcut to the \setting{Sound Settings Menu}, where you can configure volume,
bass, treble, and other settings affecting the sound of your music.  
See \reference{ref:configure_rockbox_sound} for more information.

\subsubsection{Playback Settings}
This is a shortcut to the \setting{Playback Settings Menu}, where you can configure shuffle,
repeat, party mode, skip length and other settings affecting the playback of your music.  

\subsubsection{Rating}
The menu entry is only shown if \setting{Gather Runtime Information} is
enabled. It allows the assignment of a personal rating value (0 -- 10)
to a track which can be displayed in the WPS and used in the Database
browser. The value wraps at 10.

\subsubsection{Bookmarks}
This allows you to create a bookmark in the currently-playing track.

\subsubsection{\label{ref:trackinfoviewer}Show Track Info}
\screenshot{rockbox_interface/images/ss-id3-viewer}{The track info viewer}{}
This screen is accessible from the WPS screen, and provides a detailed view of
all the identity information about the current track. This info is known as
meta data and is stored in audio file formats to keep information on artist,
album etc. To access this screen, %
\opt{RECORDER_PAD,IRIVER_H100_PAD,IRIVER_H300_PAD,IRIVER_H10_PAD,%
      SANSA_C200_PAD,SANSA_E200_PAD,SANSA_FUZE_PAD}{
  press \ActionWpsIdThreeScreen. }%
\opt{PLAYER_PAD,ONDIO_PAD,IPOD_4G_PAD,IPOD_3G_PAD,IAUDIO_X5_PAD,IAUDIO_M3_PAD,%
      GIGABEAT_PAD,GIGABEAT_S_PAD,MROBE100_PAD,SANSA_CLIP_PAD,PBELL_VIBE500_PAD,%
      MPIO_HD200_PAD}{press \ActionWpsContext{} to access the 
      \setting{WPS Context Menu} and select \setting{Show Track Info}. }
\opt{RECORDER_PAD,PLAYER_PAD,ONDIO_PAD}{Use \ButtonLeft\ and \ButtonRight\
  to move through the information.}%

\subsubsection{Open With...}
This \setting{Open With} function is the same as the \setting{Open With} 
function in the file browser's \setting{Context Menu}.

\subsubsection{Delete}
Delete the currently playing file. The file will be deleted but the playback
of the file will not stop immediately. Instead, the part of the file that
has already been buffered (i.e. read into the \daps\ memory) will be played.
This may even be the whole track.

\opt{pitchscreen}{
  \subsubsection{\label{sec:pitchscreen}Pitch}
  
  The \setting{Pitch Screen} allows you to change the rate of playback
  (i.e. the playback speed and at the same time the pitch) of your
  \dap.  The rate value can be adjusted
  between 50\% and 200\%. 50\% means half the normal playback speed and a
  pitch that is an octave lower than the normal pitch. 200\% means double
  playback speed and a pitch that is an octave higher than the normal pitch.

  The rate can be changed in two modes: procentual and semitone.
  Initially, procentual mode is active.
  
  \opt{swcodec}{
    If you've enabled the \setting{Timestretch} option in
    \setting{Sound Settings} and have since rebooted, you can also use
    timestretch mode. This allows you to change the playback speed
    without affecting the pitch, and vice versa.
    
    In timestretch mode there are separate displays for pitch and
    speed, and each can be altered independently.  Due to the
    limitations of the algorithm, speed is limited to be between 35\%
    and 250\% of the current pitch value.  Pitch must maintain the
    same ratio as well as remain between 50\% and 200\%.
  }
  
  The value of the \opt{swcodec}{rate, pitch and speed}\nopt{swcodec}{rate}
  is not persistent, i.e. after the \dap\ is turned on it will
  always be set to 100\%.  \opt{swcodec}{ However, the rate, pitch and speed
  information will be stored in any bookmarks you may create
  (see \reference{ref:Bookmarkconfigactual}) and will be restored upon
  playing back those bookmarks.}

  \nopt{swcodec}{
      \begin{btnmap}
        \ActionPsToggleMode
        & Toggle pitch changing mode. \\
        %
        \ActionPsIncSmall{} / \ActionPsDecSmall
        & Increase~/ Decrease pitch by 0.1\% (in procentual mode) or by 0.1
          semitone (in semitone mode).\\
        %
        \ActionPsIncBig{} / \ActionPsDecBig
        & Increase~/ Decrease pitch by 1\% (in procentual mode) or a semitone
          (in semitone mode).\\
        %
        \ActionPsNudgeLeft{} / \ActionPsNudgeRight
        & Temporarily change pitch by 2\% (beatmatch). \\
        %
        \ActionPsReset
        & Reset rate to 100\%. \\
        %
        \ActionPsExit
        & Leave the \setting{Pitch Screen}. \\
        %
      \end{btnmap}

    \warn{Changing the pitch can cause audible `Artifacts' or `Dropouts'.}
  }

  \opt{swcodec}{
      \begin{btnmap}
        \ActionPsToggleMode
        \opt{HAVEREMOTEKEYMAP}{& \ActionRCPsToggleMode}
        & Toggle pitch changing mode (cycle through all available modes).\\
        %
        \ActionPsIncSmall{} / \ActionPsDecSmall
        \opt{HAVEREMOTEKEYMAP}{& \ActionRCPsIncSmall{} / \ActionRCPsDecSmall}
        & Increase~/ Decrease pitch by 0.1\% (in procentual mode) or 0.1
          semitone (in semitone mode).\\
        %
        \nopt{PBELL_VIBE500_PAD}{ % there is no long scroll up or down because of slide
        \ActionPsIncBig{} / \ActionPsDecBig
        \opt{HAVEREMOTEKEYMAP}{& \ActionRCPsIncBig{} / \ActionRCPsDecBig}
        & Increase~/ Decrease pitch by 1\% (in procentual mode) or a semitone
          (in semitone mode).\\
        }
        %
        \ActionPsNudgeLeft{} / \ActionPsNudgeRight
        \opt{HAVEREMOTEKEYMAP}{& \ActionRCPsNudgeLeft{} / \ActionRCPsNudgeRight}
        & Temporarily change pitch by 2\% (beatmatch), or modify speed (in timestretch mode).\\
        %
        \ActionPsReset
        \opt{HAVEREMOTEKEYMAP}{& \ActionRCPsReset}
        & Reset pitch and speed to 100\%. \\
        %
        \ActionPsExit
        \opt{HAVEREMOTEKEYMAP}{& \ActionRCPsExit}
        & Leave the \setting{Pitch Screen}. \\
        %
      \end{btnmap}
  }

}

%********************QUICKSCREENS***********************************************
\opt{RECORDER_PAD}{
  \section{\label{ref:QuickScreens}Quick Screens}
  \screenshot{rockbox_interface/images/ss-quick-screen-112x64x1.png}{The F2 quick screen}{}
  \screenshot{rockbox_interface/images/ss-quick-screen2-112x64x1.png}{The F3 quick screen}{}
  Rockbox handles function buttons in a different way to the Archos software.
  \ButtonFOne\ is always bound to the menu function, while \ButtonFTwo\ and
  \ButtonFThree\ enable two quick screens.
  
  \ButtonFTwo\ displays some browse and play settings which are likely to be
  changed frequently. This settings are Shuffle mode, Repeat mode and the Show
  files options
  
  Shuffle mode plays each track in the currently playing list in a random order
  rather than in the order shown in the browser.

  Repeat mode repeats either a single track (One) or the entire playlist (All).

  Show files determines what type files can be seen in the browser.  This can be
  just MP3 files and directories (Music), Playlists, MP3 files and directories
  (Playlists), any files that Rockbox supports (Supported) or all files on the
  disk (All).

  See \reference{ref:PlaybackOptions} for more information about these
  settings.

  \begin{btnmap}
      \ButtonLeft & Control Shuffle mode setting. \\
      \ButtonRight & Control Repeat mode setting. \\
      \ButtonDown & Control Show file setting. \\
  \end{btnmap}
  
  \ButtonFThree\ controls frequently used display options.
  
  Scroll bar turns the display of the Scroll bar on the left of the screen on
  or off.
  
  Status bar turns the status display at the top of the screen on or off. 
  Upside down inverts the screen so that the top of the display appears nearest
  to the buttons. This is sometimes useful when storing the \dap\ in a pocket.
  Key assignments swap over with the display orientation where it is logical 
  for them to do so.

  See \reference{ref:Displayoptions} for more information about these
  settings.
  
  \begin{btnmap}
      \ButtonLeft & Control scroll bar display. \\
      \ButtonRight & Control status bar display. \\
      \ButtonDown & Control upside down screen setting. \\
  \end{btnmap}
}


%Include playlist section
% $Id$ %
\chapter{Advanced Topics}

\section{\label{ref:CustomisingUI}Customising the User Interface}
\opt{lcd_bitmap}{
\subsection{\label{ref:GettingExtras}Getting Extras}

Rockbox supports custom fonts. A collection of fonts is available for download
in the font package at \url{http://www.rockbox.org/daily.shtml}.}

\opt{lcd_bitmap}{
  \subsection{\label{ref:Loadingfonts}Loading Fonts}\index{Fonts}
  Rockbox can load fonts dynamically. Simply copy the \fname{.fnt} file to the
  \dap{} and ``play'' it in the \setting{File Browser}. If you want a font to
  be loaded automatically every time you start up, it must be located in the
  \fname{/.rockbox/fonts} directory and the filename must be at most 24 characters
  long. You can browse the fonts in \fname{/.rockbox/fonts} under
  \setting{Settings $\rightarrow$ Theme Settings $\rightarrow$ Font}
  in the \setting{Main Menu}.\\

  \note{Advanced Users Only: Any BDF font should
    be usable with Rockbox. To convert from \fname{.bdf} to \fname{.fnt}, use
    the \fname{convbdf} tool. This tool can be found in the \fname{tools}
    directory of the Rockbox source code. See \wikilink{CreateFonts\#ConvBdf}
    for more details. Or just run \fname{convbdf} without any parameters to
    see the possible options.}
}

\subsection{\label{ref:Loadinglanguages}Loading Languages}
\index{Language files}%
Rockbox can load language files at runtime. Simply copy the \fname{.lng} file 
\emph{(do not use the .lang file)} to the \dap\ and ``play'' it in the 
Rockbox directory browser or select \setting{Settings $\rightarrow$
General Settings $\rightarrow$ Language }from the \setting{Main Menu}.\\

\note{If you want a language to be loaded automatically every time you start 
up, it must be located in the \fname{/.rockbox/langs} directory and the filename
must be a maximum of 24 characters long.\\}

If your language is not yet supported and you want to write your own language
file find the instructions on the Rockbox website:
\wikilink{LangFiles}

\opt{lcd_color}{
  \subsection{\label{ref:ChangingFiletypeColours}Changing Filetype Colours}
  Rockbox has the capability to modify the \setting{File Browser} to show
  files of different types in different colours, depending on the file extension. 

  \subsubsection{Set-up}
  There are two steps to changing the filetype colours -- creating
  a file with the extension \fname{.colours} and then activating it using
  a config file.  The \fname{.colours} files \emph{must} be stored in
  the \fname{/.rockbox/themes/} directory.
  The \fname{.colours} file is just a text file, and can be edited with
  your text editor of choice.

  \subsubsection{Creating the .colours file}
  The \fname{.colours} file consists of the file extension
  (or \fname{folder}) followed by a colon and then the colour desired
  as an RGB value in hexadecimal, as in the following example:\\*
  \\
  \config{folder:808080}\\
  \config{mp3:00FF00}\\
  \config{ogg:00FF00}\\
  \config{txt:FF0000}\\
  \config{???:FFFFFF}\\*

  The permissible extensions are as follows:\\*
  \\ 
  \config{folder, m3u, m3u8, cfg, wps, lng, rock, bmark, cue, colours, mpa,
  \firmwareextension{}, %
  \opt{swcodec}{mp1, }mp2, mp3%
  \opt{swcodec}{, ogg, oga, wma, wmv, asf, wav, flac, ac3, a52, mpc,
  wv, m4a, m4b, mp4, mod, shn, aif, aiff, spx, sid, adx, nsf, nsfe,
  spc, ape, mac, sap}%
  \opt{lcd_bitmap}{\opt{swcodec}{, mpg, mpeg}}%
  \opt{HAVE_REMOTE_LCD}{, rwps}%
  \opt{lcd_non-mono}{, bmp}%
  \opt{radio}{, fmr}%
  \opt{lcd_bitmap}{, fnt, kbd}}\\*
  %It'd be ideal to get these from filetypes.c

  All file extensions that are not either specifically listed in the
  \fname{.colours} files or are not in the list above will be
  set to the colour given by \config{???}. Extensions that
  are in the above list but not in the \fname{.colours}
  file will be set to the foreground colour as normal.

  \subsubsection{Activating}
  To activate the filetype colours, the \fname{.colours} file needs to be
  invoked from a \fname{.cfg} configuration file. The easiest way to do
  this is to create a new text file containing the following single
  line:\\*
  \\
  \config{filetype colours: /.rockbox/themes/filename.colours}\\*

  where filename is replaced by the filename you used when creating the
  \fname{.colours} file. Save this file as e.g. \fname{colours.cfg} in the
  \fname{/.rockbox/themes} directory and then activate the config file
  from the menu as normal
  (\setting{Settings} $\rightarrow$ \setting{Theme Settings}%
  $\rightarrow$ \setting{Browse Theme Files}).

  \subsubsection{Editing}
  The built-in \setting{Text Editor} (see \reference{sec:text_editor})
  automatically understands the
  \fname{.colours} file format, but an external text editor can
  also be used. To edit the \fname{.colours} file using Rockbox,
  ``play'' it in the \setting{File Browser}. The file will open in 
  the \setting{Text Editor}. Upon selecting a line, the following choices
  will appear:\\*
  \\
  \config{Extension}\\
  \config{Colour}\\*

  If \config{Extension} is selected, the \setting{virtual keyboard}
  (see \reference{sec:virtual_keyboard}) appears,
  allowing the file extension to be modified. If \config{Colour}
  is selected, the colour selector screen appears. Choose the desired
  colour, then save the \fname{.colours} file using the standard
  \setting{Text Editor} controls.
}

\opt{lcd_non-mono}{%
  \subsection{\label{ref:LoadingBackdrops}Loading Backdrops}
  Rockbox supports showing an image as a backdrop in the \setting{File Browser}
  and the menus. The backdrop image must be a \fname{.bmp} file of the exact
  same dimensions as the display in your \dap{} (\dapdisplaysize{} with the last
  number giving the colour depth in bits). To use an image as a backdrop browse
  to it in the \setting{File Browser} and open the \setting{Context Menu}
  (see \reference{ref:Contextmenu}) on it and select the option
  \setting{Set As Backdrop}. If you want rockbox to remember your
  backdrop the next time you start your \dap{} the backdrop must be placed in
  the \fname{/.rockbox/backdrops} directory.
}%

\nopt{lcd_charcell}{
  \subsection{UI Viewport}
  By default, the UI is drawn on the whole screen. This can be changed so that
  the UI is confined to a specific area of the screen, by use of a UI
  viewport. This is done by adding the following line to the 
  \fname{.cfg} file for a theme:\\*

  \nopt{lcd_non-mono}{\config{ui viewport: X,Y,[width],[height],[font]}}
  \nopt{lcd_color}{\opt{lcd_non-mono}{
      \config{ui viewport: X,Y,[width],[height],[font],[fgshade],[bgshade]}}}
  \opt{lcd_color}{
      \config{ui viewport: X,Y,[width],[height],[font],[fgcolour],[bgcolour]}}
  \\*

  \opt{HAVE_REMOTE_LCD}{
    The dimensions of the menu that is displayed on the remote control of your
    \dap\ can be set in the same way.  The line to be added to the theme
    \fname{.cfg} is the following:\\*

    \nopt{lcd_non-mono}{\config{remote ui viewport: X,Y,[width],[height],[font]}}
    \nopt{lcd_color}{\opt{lcd_non-mono}{
      \config{remote ui viewport: X,Y,[width],[height],[font],[fgshade],[bgshade]}}}
    \opt{lcd_color}{
      \config{remote ui viewport: X,Y,[width],[height],[font],[fgcolour],[bgcolour]}}
  \\*
  }

  Only the first two parameters \emph{have} to be specified, the others can
  be omitted using `-' as a placeholder. The syntax is very similar to WPS 
  viewports (see \reference{ref:Viewports}).  Briefly:

  \nopt{lcd_non-mono}{\input{advanced_topics/viewports/mono-uivp-syntax.tex}}
  \nopt{lcd_color}{\opt{lcd_non-mono}{\input{advanced_topics/viewports/grayscale-uivp-syntax.tex}}}
  \opt{lcd_color}{\input{advanced_topics/viewports/colour-uivp-syntax.tex}}
}

\section{\label{ref:ConfiguringtheWPS}Configuring the Theme}

\subsection{Themeing -- General Info}

  There are various different aspects of the Rockbox interface
  that can be themed -- the WPS or \setting{While Playing Screen}, the FMS or
  \setting{FM Screen} (if the \dap{} has a tuner), and the SBS or
  \setting{Base Skin}. The WPS is the name used to
  describe the information displayed on the \daps{} screen whilst an audio
  track is being played, the FMS is the screen shown while listening to the
  radio, and the SBS lets you specify a base skin that is shown in the
  menus and browsers, as well as the WPS and FMS. The SBS also allows you to
  control certain aspects of the appearance of the menus/browsers.
  There are a number of themes included in Rockbox, and
  you can load one of these at any time by selecting it in
  \setting{Settings $\rightarrow$ Theme Settings $\rightarrow$ Browse Theme Files}.
  It is also possible to set individual items of a theme from within the
  \setting{Settings $\rightarrow$ Theme Settings} menu.


\subsection{\label{ref:CreateYourOwnWPS}Themes -- Create Your Own}
The theme files are simple text files, and can be created (or edited) in your
favourite text editor. To make sure non-English characters 
display correctly in your theme you must save the theme files with UTF-8
character encoding. This can be done in most editors, for example Notepad in
Windows 2000 or XP (but not in 9x/ME) can do this.

\begin{description}
\item [Files Locations: ] Each different ``themeable'' aspect requires its own file --
  WPS files have the extension \fname{.wps}, FM screen files have the extension
  \fname{.fms}, and SBS files have the extension \fname{.sbs}. The main theme
  file has the extension \fname{.cfg}. All files should have the same name.
  
  The theme \fname{.cfg} file should be placed in the \fname{/.rockbox/themes}
  directory, while the \fname{.wps}, \fname{.fms} and \fname{.sbs} files should
  be placed in the \fname{/.rockbox/wps} directory. Any images used by the
  theme should be placed in a subdirectory of \fname{/.rockbox/wps} with the
  same name as the theme, e.g. if the theme files are named
  \fname{mytheme.wps, mytheme.sbs} etc., then the images should be placed in
  \fname{/.rockbox/wps/mytheme}.
\end{description}

All full list of the available tags are given in appendix
\reference{ref:wps_tags}; some of the more powerful concepts in theme design
are discussed below.

\begin{itemize}
\item All characters not preceded by \% are displayed as typed.
\item Lines beginning with \# are comments and will be ignored.
\end{itemize}

\note{Keep in mind that your \daps{} resolution is \dapdisplaysize{} (with
  the last number giving the colour depth in bits) when
  designing your own WPS, or if you use a WPS designed for another target.
  \opt{HAVE_REMOTE_LCD}{The resolution of the remote is
      \opt{iriverh100,iriverh300}{128$\times$64$\times$1}%
      \opt{iaudiox5,iaudiom5,iaudiom3}{128$\times$96$\times$2}
      pixels.
  }
}

\nopt{lcd_charcell}{
\subsubsection{\label{ref:Viewports}Viewports}

By default, a viewport filling the whole screen contains all the elements
defined in each theme file. The 
\opt{lcd_non-mono}{elements in this viewport are displayed
  with the same background/\linebreak{}foreground
  \opt{lcd_color}{colours}\nopt{lcd_color}{shades} and the}
text is rendered in the
same font as in the main menu. To change this behaviour a custom viewport can
be defined. A viewport is a rectangular window on the screen% 
\opt{lcd_non-mono}{ with its own foreground/background
\opt{lcd_color}{colours}\nopt{lcd_color}{shades}}.
This window also has variable dimensions. To
define a viewport a line starting \config{{\%V(\dots}} has to be
present in the theme file. The full syntax will be explained later in
this section. All elements placed before the 
line defining a viewport are displayed in the default viewport. Elements
defined after a viewport declaration are drawn within that viewport.
\opt{lcd_bitmap}{Loading images (see Appendix \reference{ref:wps_images})
  should be done within the default viewport.}
A viewport ends either with the end of the file, or with the next viewport
declaration line. Viewports sharing the same
coordinates and dimensions cannot be displayed at the same time. Viewports
cannot be layered \emph{transparently} over one another. Subsequent viewports
will be drawn over any other viewports already drawn onto that
area of the screen.

\nopt{lcd_non-mono}{\input{advanced_topics/viewports/mono-vp-syntax.tex}}
\nopt{lcd_color}{\opt{lcd_non-mono}{\input{advanced_topics/viewports/grayscale-vp-syntax.tex}}}
\opt{lcd_color}{\input{advanced_topics/viewports/colour-vp-syntax.tex}}


\subsubsection{Conditional Viewports}

Any viewport can be displayed either permanently or conditionally.
Defining a viewport as \config{{\%V(\dots}}
will display it permanently.

\begin{itemize}
\item {\config{\%Vl('identifier',\dots)}}
This tag preloads a viewport for later display. `identifier' is a single
lowercase letter (a-z) and the `\dots' parameters use the same logic as
the \config{\%V} tag explained above.
\item {\config{\%Vd('identifier')}} Display the `identifier' viewport.
\end{itemize}

Viewports can share identifiers so that you can display multiple viewports
with one \%Vd line.

\nopt{lcd_non-mono}{\input{advanced_topics/viewports/mono-conditional.tex}}
\nopt{lcd_color}{%
  \opt{lcd_non-mono}{\input{advanced_topics/viewports/grayscale-conditional.tex}}}
\opt{lcd_color}{\input{advanced_topics/viewports/colour-conditional.tex}}
\\*

\note{The tag to display conditional viewports must come before the tag to
preload the viewport in the \fname{.wps} file.}

\subsection{Info Viewport (SBS only)}
As mentioned above, it is possible to set a UI viewport via the theme
\fname{.cfg} file. It is also possible to set the UI viewport through the SBS
file, and to conditionally select different UI viewports.

  \begin{itemize}
    \item {\config{\%Vi('label',\dots)}}
    This viewport is used as Custom UI Viewport in the case that the theme
    doesn't have a ui viewport set in the theme \fname{.cfg} file. Having this
    is strongly recommended since it makes you able to use the SBS
    with other themes. If label is set this viewport can be selectivly used as the
    Info Viewport using the \%VI tag. The `\dots' parameters use the same logic as
    the \config{\%V} tag explained above.

    \item {\config{\%VI('label')}} Set the Info Viewport to use the viewport called
    label, as declared with the previous tag.
  \end{itemize}

\subsection{\label{ref:multifont}Additional Fonts}
Additional fonts can be loaded within each screen file to be used in that
screen. In this way not only can you have different fonts between e.g. the menu
and the WPS, but you can use multiple fonts in each of the individual screens.\\

\config{\%Fl('id',filename,glyphs)}

  \begin{itemize}
    \item `id' is the number you want to use in viewport declarations, 0 and 1
       are reserved and so can't be used.
    \item `filename' is the font filename to load. Fonts should be stored in
       \fname{/.rockbox/fonts/}
    \item `glyphs' is an optional specification of how many unique glyphs to
       store in memory. Default is 256.
  \end{itemize}

  An example would be: \config{\%Fl(2,12-Nimbus.fnt,100)}

}

\subsubsection{Conditional Tags}

\begin{description}
\item[If/else: ]
Syntax: \config{\%?xx{\textless}true{\textbar}false{\textgreater}}

If the tag specified by ``\config{xx}'' has a value, the text between the 
``\config{{\textless}}'' and the ``\config{{\textbar}}'' is displayed (the true
part), else the text between the ``\config{{\textbar}}'' and the 
``\config{{\textgreater}}'' is displayed (the false part).
The else part is optional, so the ``\config{{\textbar}}'' does not have to be 
specified if no else part is desired. The conditionals nest, so the text in the
if and else part can contain all \config{\%} commands, including conditionals.

\item[Enumerations: ]
Syntax: \config{\%?xx{\textless}alt1{\textbar}alt2{\textbar}alt3{\textbar}\dots{\textbar}else{\textgreater}}

For tags with multiple values, like Play status, the conditional can hold a 
list of alternatives, one for each value the tag can have.
Example enumeration: 
\begin{example}
     \%?mp{\textless}Stop{\textbar}\%Play{\textbar}Pause{\textbar}Ffwd{\textbar}Rew{\textgreater}
\end{example}

The last else part is optional, and will be displayed if the tag has no value. 
The WPS parser will always display the last part if the tag has no value, or if
the list of alternatives is too short.
\end{description}

\subsubsection{Next Song Info}
You can display information about the next song -- the song that is
about to play after the one currently playing (unless you change the
plan).

If you use the upper-case versions of the
three tags: \config{F}, \config{I} and \config{D}, they will instead refer to 
the next song instead of the current one. Example: \config{\%Ig} is the genre 
name used in the next song and \config{\%Ff} is the mp3 frequency.\\

\note{The next song information \emph{will not} be available at all
  times, but will most likely be available at the end of a song. We
  suggest you use the conditional display tag a lot when displaying
  information about the next song!}

\subsubsection{\label{ref:AlternatingSublines}Alternating Sublines}

It is possible to group items on each line into 2 or more groups or 
``sublines''. Each subline will be displayed in succession on the line for a 
specified time, alternating continuously through each defined subline.

Items on a line are broken into sublines with the semicolon
`\config{;}' character. The display time for
each subline defaults to 2 seconds unless modified by using the
`\config{\%t}' tag to specify an alternate
time (in seconds and optional tenths of a second) for the subline to be
displayed. 

Subline related special characters and tags: 
\begin{description}
\item[;] Split items on a line into separate sublines
\item[\%t] Set the subline display time. The
`\config{\%t}' is followed by either integer seconds (\config{\%t5}), or seconds
and tenths of a second within () e.g. (\config{\%t(3.5)}).
\end{description}

Each alternating subline can still be optionally scrolled while it is
being displayed, and scrollable formats can be displayed on the same
line with non{}-scrollable formats (such as track elapsed time) as long
as they are separated into different sublines.
Example subline definition:
\begin{example}
     %s%t(4)%ia;%s%it;%t(3)%pc %pr : Display id3 artist for 4 seconds,
                                 Display id3 title for 2 seconds,
                                 Display current and remaining track time
                                 for 3 seconds,
                                 repeat...
\end{example}

Conditionals can be used with sublines to display a different set and/or number
of sublines on the line depending on the evaluation of the conditional.
Example subline with conditionals:
\begin{example}
    %?it{\textless}%t(8)%s%it{\textbar}%s%fn{\textgreater};%?ia{\textless}%t(3)%s%ia{\textbar}%t(0){\textgreater}\\
\end{example}

The format above will do two different things depending if ID3 tags are 
present. If the ID3 artist and title are present:
\begin{itemize}
\item Display id3 title for 8 seconds,
\item Display id3 artist for 3 seconds,
\item repeat\dots
\end{itemize}
If the ID3 artist and title are not present:
\begin{itemize}
\item Display the filename continuously.
\end{itemize}
Note that by using a subline display time of 0 in one branch of a conditional,
a subline can be skipped (not displayed) when that condition is met. 

\subsubsection{Using Images}
You can have as many as 52 images in your WPS. There are various ways of 
displaying images:
\begin{enumerate}
  \item Load and always show the image, using the \config{\%x} tag
  \item Preload the image with \config{\%xl} and show it with \config{\%xd}. 
    This way you can have your images displayed conditionally.
    \nopt{archos}{%
    \item Load an image and show as backdrop using the \config{\%X} tag. The 
      image must be of the same exact dimensions as your display.
    }%
\end{enumerate}

\optv{swcodec}{% This doesn't depend on swcodec but we don't have a \noptv
               % command.
  Example on background image use:
  \begin{example}
    %X(background.bmp)
  \end{example}
  The image with filename \fname{background.bmp} is loaded and used in the WPS.
}%

Example on bitmap preloading and use:
\begin{example}
    %x(a,static_icon.bmp,50,50)
    %xl(b,rep\_off.bmp,16,64)
    %xl(c,rep\_all.bmp,16,64)
    %xl(d,rep\_one.bmp,16,64)
    %xl(e,rep\_shuffle.bmp,16,64)
    %?mm<%xd(b)|%xd(c)|%xd(d)|%xd(e)>
\end{example}
Four images at the same x and y position are preloaded in the example. Which 
image to display is determined by the \config{\%mm} tag (the repeat mode).

\subsubsection{Example File}
\begin{example}
    %s%?in<%in - >%?it<%it|%fn> %?ia<[%ia%?id<, %id>]> 
    %pb%pc/%pt
\end{example}
That is, ``tracknum -- title [artist, album]'', where most fields are only
displayed if available. Could also be rendered as ``filename'' or ``tracknum --
title [artist]''.

%\opt{lcd_bitmap}{
%  \begin{verbatim}
%    %s%?it<%?in<%in. |>%it|%fn>
%    %s%?ia<%ia|%?d2<%d(2)|(root)>>
%    %s%?id<%id|%?d1<%d(1)|(root)>> %?iy<(%iy)|>
%  
%    %al%pc/%pt%ar[%pp:%pe]
%    %fbkBit %?fv<avg|> %?iv<(id3v%iv)|(no id3)>
%    %pb
%    %pm
% % \end{verbatim}
%}

\section{\label{ref:manage_settings}Managing Rockbox Settings}

\subsection{Introduction to \fname{.cfg} Files}
Rockbox allows users to store and load multiple settings through the use of
configuration files. A configuration file is simply a text file with the
extension \fname{.cfg}.

A configuration file may reside anywhere on the disk. Multiple
configuration files are permitted. So, for example, you could have
a \fname{car.cfg} file for the settings that you use while playing your
jukebox in your car, and a \fname{headphones.cfg} file to store the
settings that you use while listening to your \dap{} through headphones.

See \reference{ref:cfg_specs} below for an explanation of the format 
for configuration files. See \reference{ref:manage_settings_menu} for an
explanation of how to create, edit and load configuration files.

\subsection{\label{ref:cfg_specs}Specifications for \fname{.cfg} Files}

The Rockbox configuration file is a plain text file, so once you use the 
\setting{Save .cfg file} option to create the file, you can edit the file on 
your computer using any text editor program. See
Appendix \reference{ref:config_file_options} for available settings. Configuration 
files use the following formatting rules: %

\begin{enumerate} 
\item Each setting must be on a separate line. 
\item Each line has the format ``setting: value''. 
\item Values must be within the ranges specified in this manual for each 
  setting. 
\item Lines starting with \# are ignored. This lets you write comments into 
  your configuration files. 
\end{enumerate}

Example of a configuration file:
\begin{example}
    volume: 70
    bass: 11
    treble: 12
    balance: 0
    time format: 12hour
    volume display: numeric
    show files: supported
    wps: /.rockbox/car.wps
    lang: /.rockbox/afrikaans.lng
\end{example}

\note{As you can see from the example, configuration files do not need to 
  contain all of the Rockbox options.  You can create configuration files 
  that change only certain settings. So, for example, suppose you 
  typically use the \dap{} at one volume in the car, and another when using 
  headphones. Further, suppose you like to use an inverse LCD when you are 
  in the car, and a regular LCD setting when you are using headphones. You 
  could create configuration files that control only the volume and LCD 
  settings. Create a few different files with different settings, give 
  each file a different name (such as \fname{car.cfg}, 
  \fname{headphones.cfg}, etc.), and you can then use the \setting{Browse .cfg 
    files} option to quickly change settings.\\} 

  A special case configuration file can be used to force a particular setting
  or settings every time Rockbox starts up (e.g. to set the volume to a safe
  level). Format a new configuration file as above with the required setting(s)
  and save it into the \fname{/.rockbox} directory with the filename
  \fname{fixed.cfg}.

\subsection{\label{ref:manage_settings_menu}The \setting{Manage Settings} 
  menu} The \setting{Manage Settings} menu can be found in the \setting{Main 
  Menu}. The \setting{Manage Settings} menu allows you to save and load 
  \fname{.cfg} files.

\begin{description}
  
\item [Browse .cfg Files]Opens the \setting{File Browser} in the
  \fname{/.rockbox} directory and displays all \fname{.cfg} (configuration)
  files. Selecting a \fname{.cfg} file will cause Rockbox to load the settings
  contained in that file. Pressing \ActionStdCancel{} will exit back to the
  \setting{Manage Settings} menu. See the \setting{Write .cfg files} option on
  the \setting{Manage Settings} menu for details of how to save and edit a 
  configuration file.
  
\item [Reset Settings]This wipes the saved settings
  in the \dap{} and resets all settings to their default values. 
  
  \opt{IRIVER_H100_PAD,IRIVER_H300_PAD,IAUDIO_X5_PAD,SANSA_E200_PAD,SANSA_C200_PAD%
      ,PBELL_VIBE500_PAD}{
      \note{You can also reset all settings to their default 
      values by turning off the \dap, turning it back on, and holding the
      \ButtonRec{} button immediately after the \dap{} turns on.}
  } 
  \opt{IRIVER_H10_PAD}{\note{You can also reset all settings to
      their default values by turning off the \dap, and turning it back on
      with the \ButtonHold{} button on.}
  }
  \opt{IPOD_4G_PAD}{\note{You can also reset all settings to their default 
      values by turning off the \dap, turning it back on, and activating the
      \ButtonHold{} button immediately after the backlight comes on.}
  }
  \opt{GIGABEAT_PAD}{\note{You can also reset all settings to their default
      values by turning off the \dap, turning it back on and pressing the
      \ButtonA{} button immediately after the \dap{} turns on.}
  }

\item [Save .cfg File]This option writes a \fname{.cfg} file to 
  your \daps{} disk. The configuration file has the \fname{.cfg} 
  extension and is used to store all of the user settings that are described 
  throughout this manual.

  Hint: Use the \setting{Save .cfg File} feature (\setting{Main Menu 
    $\rightarrow$ Manage Settings}) to save the current settings, then 
  use a text editor to customize the settings file. See Appendix 
  \reference{ref:config_file_options} for the full reference of available 
  options.
  
\item [Save Sound Settings]This option writes a \fname{.cfg} file to 
  your \daps{} disk. The configuration file has the \fname{.cfg} 
  extension and is used to store all of the sound related settings.
    
\item [Save Theme Settings]This option writes a \fname{.cfg} file to 
  your \daps{} disk. The configuration file has the \fname{.cfg} 
  extension and is used to store all of the theme related settings.

\end{description}

\section{\label{ref:FirmwareLoading}Firmware Loading}
\opt{player,recorder,recorderv2fm,ondio}{
  When your \dap{} powers on, it loads the Archos firmware in ROM, which
  automatically checks your \daps{} root directory for a file named 
  \firmwarefilename. Note that Archos firmware can only read the first 
  ten characters of each filename in this process, so do not rename your old 
  firmware files with names like \firmwarefilename.\fname{old} and so on, 
  because it is possible that the \dap{} will load a file other than the one 
  you intended.
}

\subsection{\label{ref:using_rolo}Using ROLO (Rockbox Loader)}
Rockbox is able to load and start another firmware file without rebooting. 
You just ``play'' a file with the extension %
\opt{recorder,recorderv2fm,ondio}{\fname{.ajz}.} %
\opt{player}{\fname{.mod}.} %
\opt{iriverh100,iriverh300}{\fname{.iriver}.} %
\opt{ipod}{\fname{.ipod}.} %
\opt{iaudio}{\fname{.iaudio}.} %
\opt{sansa,iriverh10,iriverh10_5gb,vibe500}{\fname{.mi4}.} %
\opt{sansaAMS}{\fname{.sansa}.} %
\opt{gigabeatf,gigabeats}{\fname{.gigabeat}.} %
This can be used to test new firmware versions without deleting your
current version.

\opt{archos}{\section{\label{ref:Rockboxinflash}Rockbox in Flash}

\subsection{Introduction}

When you bought your \playertype, it came with the \playerman\ firmware in
flash ROM. When you power on your \dap, this \playerman\ firmware starts,
and then loads an updated firmware from disk if present (\firmwarefilename).
An ordinary Rockbox installation only replaces the on-disk firmware, leaving
the flash ROM contents intact. That means the \playerman\ firmware still
controls the boot process.

The main reason to change this is to improve the startup time of your player.
The \playerman\ bootloader is rather slow. With Rockbox in flash, your \dap\
will boot much faster, typically in three to five seconds. Furthermore you
might prefer a clean Rockbox environment, with as little remnants of the
\playerman\ software as possible.
\opt{rombox}{On your \dap\ it is also possible to execute Rockbox directly
  from flash ROM, increasing the amount of free RAM for buffering music. This
  is called \emph{Rombox}.
}

\warn{Flashing your \dap\ is somewhat dangerous, like programming a mainboard
  \emph{BIOS}, \emph{CD/DVD} drive firmware, mobile phone, etc. If the power
  fails, the chip breaks while programming or most of all the programming
  software malfunctions, you'll have a dead box. We take no responsibility of
  any kind, you do that at your own risk. However, we tried as carefully as
  possible to bulletproof this code. There are a lot of sanity checks. If any
  of them fails, it will not program.
}

\opt{ondio}{\warn{After flashing Rockbox, never try to ROLO the \playerman\
    firmware
    \opt{ondiofm}{versions 1.31f or 1.32b! These versions are flash updates
      themselves. If they are}
    \opt{ondiosp}{version 1.32b! This version is a flash update itself.
      If it is}
    applied when Rockbox is flashed, you'll end up with a garbled flash ROM
    and hence a dead box.
}}

There's an ultimate safety net to bring back boxes with even completely
garbled flash content: the \emph{UART} boot mod, which in turn requires the
\emph{serial} mod. With that it's possible to reflash independently from the
outside, even if the flash ROM is completely erased.
\nopt{ondio}{This won't work if you have one of the rare ``ROMless'' boxes. These
  have no boot ROM and boot directly from flash.
}
If the first $\approx$2~KB of the flash ROM are flashed OK, \emph{Minimon} can
be used for the same purpose.

\subsection{Terminology and Basic Operation}

\begin{description}
\item[Firmware:] The flash ROM contents as a whole.
\item[Image:] One operating software started from there.
\end{description}

The replacement firmware contains a bootloader and two images. The first image
is the \emph{permanent} rescue software, to be used in case something is wrong
with the second (main) image. In current firmware files this first image
contains \emph{Bootbox} (see wiki for details). The second image is what is
booted by default. The current firmware files contain a copy of Rockbox 3.2
in the main image. It can easily be updated/replaced later.

The bootloader allows to select which image to run. Pressing
\opt{RECORDER_PAD}{\ButtonFOne}\opt{PLAYER_PAD,ONDIO_PAD}{\ButtonLeft} at boot
selects the first image.
\opt{RECORDER_PAD}{\ButtonFTwo}\opt{PLAYER_PAD}{\ButtonPlay}\opt{ONDIO_PAD}{\ButtonUp}
selects the second image, which will also be booted if you don't press any
button. The button mapping is only there for completeness.
\opt{RECORDER_PAD}{\ButtonFThree}\opt{PLAYER_PAD,ONDIO_PAD}{\ButtonRight}
selects the built-in serial monitor called \emph{Minimon}. You should know this
in case you invoke it by accident. Minimon won't display anything on the
screen. To get out of it, perform a hardware shutdown of your \dap.

\subsection{Initial Flashing Procedure}

You only need to perform this procedure the first time you flash your
\playertype. You may also want to perform it in case the update procedure for
the second image recommends it. In the latter case do not perform the steps
listed under ``Preparation''.

\subsubsection{Preparation}

\nopt{ondio}{
  First, check whether your \playertype\ is flashable at all. Select
  \setting{System $\rightarrow$ Debug (Keep Out!) $\rightarrow$ View HW
  Info}. 
  \opt{lcd_charcell}{Cycle through the displayed values with \ButtonRight /
    \ButtonLeft\ until ``Flash:'' is displayed. If it shows question marks,
  }
  \opt{lcd_bitmap}{Check the values in the line starting with ``Flash:''. If it
    shows question marks after ``M='' and ``D='',
  }
  you're out of luck, your \dap\ is not flashable without modifying the
  hardware. You can stop here. Sorry.
}

\nopt{ondio}{If your \dap\ is flashable, you}\opt{ondio}{You} should perform a
backup of the current flash ROM contents, in case you want to restore it later.
Select \setting{System $\rightarrow$ Debug (Keep Out!) $\rightarrow$ Dump ROM
contents}. You'll notice a few seconds of disk activity. When you connect your
\dap\ to the PC afterwards, you'll find two files in the root of your \dap.
Copy the 256~KB-sized file named \fname{internal\_rom\_2000000-203FFFF.bin} to
a safe place.

\subsubsection{Flashing}

\begin{enumerate}
\item Download the correct package for your \dap\ from
  \url{http://download.rockbox.org/bootloader/archos/}. It is named
  \fname{flash-{\textless}model{\textgreater}-{\textless}version{\textgreater}.zip}.
  The current packages are v3.
\item Unzip the flash package to the root of your \dap.
  \nopt{ondio}{This will extract two files to the root,
    \fname{firmware\_{\textless}model{\textgreater}.bin} and
    \fname{firmware\_{\textless}model{\textgreater}\_norom.bin}.
    \opt{recorder,recorderv2fm}{(The {\textless}model{\textgreater} part is
      slighty different from that in the .zip file name.)
    }
    The flash plugin will select the correct one for your \dap.
  }
  \opt{ondio}{This will extract one file to the root,
    \fname{firmware\_{\textless}model{\textgreater}.bin}.
  }
  Now safely disconnect USB.
\item
  \nopt{ondio}{Make sure your batteries are in good shape and fully charged.}
  \opt{ondio}{Make sure you use a set of fresh batteries.}
  Flashing doesn't need more power than normal operation, but you don't want
  your \dap\ to run out of power while flashing.
\item Select \setting{Plugins $\rightarrow$ Applications}, and run the
  \fname{firmware\_flash} plugin. It will tell you about your flash and
  which file it is going to program. After pressing
  \opt{RECORDER_PAD}{\ButtonFOne}\opt{PLAYER_PAD}{\ButtonMenu}\opt{ONDIO_PAD}{\ButtonLeft}
  it will check the file. If the file is OK, pressing
  \opt{RECORDER_PAD}{\ButtonFTwo}\opt{PLAYER_PAD}{\ButtonOn}\opt{ONDIO_PAD}{\ButtonUp}
  will give you a big warning. If we still didn't manage to scare you off, you
  need to press 
  \opt{RECORDER_PAD}{\ButtonFThree}\opt{PLAYER_PAD,ONDIO_PAD}{\ButtonRight}
  to actually program and verify. The programming takes just a few seconds.
\item In the unlikely event that the programming or verify steps should give
  you any error, \emph{do not switch off the box!} Otherwise you'll have seen
  it working for the last time. While Rockbox is still in RAM and operational,
  we could upgrade the plugin via USB and try again. If you switch it off,
  it's gone.
\end{enumerate}

\note{After successful flashing you may delete the \fname{.bin} files from the
  root of your \dap.
}

\note{There are no separate flash packages for {\dap}s modified to have 8~MB
  of RAM. You need to use the corresponding package for non-modified
  \playertype. You should then install a Rockbox image that makes use of all
  available RAM as described in the following section.
}

\subsection{Updating the Rockbox Image in Flash}

When Rockbox is booted from flash, it does not check for an updated firmware
on disk. This is one of the reasons why it boots faster than the \playerman\
firmware. It means that whenever you update Rockbox, you also need to update
the image in the flash. This is a simple and safe procedure:

\begin{enumerate}
\item Download (or build) the Rockbox build you want to use, and unzip it to
  the root of your \dap. Safely disconnect USB.
\item ROLO into the new Rockbox version.
\item Go to the file browser, and enter the \fname{.rockbox} directory (you
  might need to set the \setting{File View} option to \setting{All}.)
\item Play the file \fname{rockbox.ucl}\opt{rombox}{, or preferably
  \fname{rombox.ucl}}, and follow the instructions. The plugin handling
  this is \fname{rockbox\_flash}, a viewer plugin.
\end{enumerate}

\subsection{Restoring the Original Flash ROM Contents}

In case you ever want to restore the original flash contents, you will need
the backup file. The procedure is very similar to initial flashing, with the
following differences:

\begin{enumerate}
\item Check that you do not have any \fname{firmware\_*.bin} files in your
  \dap's root.
\item Select \setting{Plugins $\rightarrow$ Applications}, and run the
  \fname{firmware\_flash} plugin. Write down the filename it displays in the
  first screen, then exit the plugin.
\item Connect USB, and copy the flash ROM backup file to the root of your
  \dap. \emph{Only use the backup file from that very box, otherwise you're
  asking for trouble!} Rename the file so that it matches the name requested
  by the \fname{firmware\_flash} plugin. Safely disconnect USB.
\end{enumerate}

Now follow the instructions given for initial flashing, starting with step 3.
}

\section{Optimising battery runtime}
  Rockbox offers a lot of settings that have high impact on the battery runtime 
  of your \dap{}. The largest power savings can be achieved through disabling
  unneeded hardware components -- for some of those there are settings
  available. 
\opt{swcodec}{
  Another area of savings is avoiding or reducing CPU boosting
  through disabling computing intense features (e.g. sound processing) or 
  using effective audio codecs. 
} The following provides a short overview of the most relevant settings and 
  rules of thumb.

\nopt{ondio}{
\subsection{Display backlight}
  The active backlight consumes a lot of power. Therefore choose a setting that
  disables the backlight after timeout (for setting \setting{Backlight} see 
  \reference{ref:Displayoptions}). Avoid to have the backlight enabled all the 
  time.
}

\opt{lcd_sleep}{
\subsection{Display power-off}
  Shutting down the display and the display controller saves a reasonable amount 
  of power. Choose a setting that will put the display to sleep after timeout
  (for setting \setting{Sleep} see \reference{ref:Displayoptions}). Avoid to 
  have the display enabled all the time -- even, if the display is transflective
  and is readable without backlight. Depending on your \dap{} it might be 
  significantly more efficient to re-enable the display and its backlight for a 
  glimpse a few times per hour than to keep the display enabled.
}

\opt{accessory_supply}{
\subsection{Accessory power supply}
  As default your \dap{}'s accessory power supply is always enabled to ensure
  proper function of connected accessory devices. Disable this power supply, if 
  -- or as long as -- you do not use any accessory device with your \dap{} while 
  running Rockbox (see \reference{ref:AccessoryPowerSupply}).
}

\opt{lineout_poweroff}{
\subsection{Line Out}
  Rockbox allows to switch off the line-out on your \dap{}. If you do not need 
  the line-out, switch it off (see \reference{ref:LineoutOnOff}).
}

\opt{spdif_power}{
\subsection{Optical Output}
  Rockbox allows to switch off the S/PDIF output on your \dap{}. If you do not 
  need this output, switch it off (see \reference{ref:SPDIF_OnOff}).
}

\opt{disk_storage}{
\subsection{Anti-Skip Buffer}
  Having a large anti-skip buffer tends to use more power, and may reduce your
  battery life. It is recommended to always use the lowest possible setting 
  that allows correct and continuous playback (see \reference{ref:AntiSkipBuf}).
}

\opt{swcodec}{
\subsection{Replaygain}
  Replaygain is a post processing that equalises the playback volume of audio 
  files to the same perceived loudness. This post processing applies a factor 
  to each single PCM sample and is therefore consuming additional CPU time. If 
  you want to achieve some (minor) savings in runtime, switch this feature off 
  (see \reference{ref:ReplayGain}).
}

\opt{swcodec,disk_storage,flash_storage}{
\subsection{Audio format and bitrate}
\opt{swcodec}{
  In general the fastest decoding audio format will be the best in terms of
  battery runtime on your \dap{}. An overview of different codec's performance 
  on different \dap{}s can be found at \wikilink{CodecPerformanceComparison}.
}

\opt{flash_storage}{
  Your target uses flash that consumes a certain amount of power during access.
  The less often the flash needs to be switched on for buffering and the shorter 
  the buffering duration is, the lower is the overall power consumption. 
  Therefore the bitrate of the audio files does have an impact on the battery 
  runtime as well. Lower bitrate audio files will result in longer battery 
  runtime.
}
\opt{disk_storage}{
  Your target uses a hard disk which consumes a large amount of power while
  spinning -- up to several hundred mA. The less often the hard disk needs to 
  spin up for buffering and the shorter the buffering duration is, the lower is 
  the power consumption. Therefore the bitrate of the audio files does have an 
  impact on the battery runtime as well. Lower bitrate audio files will result 
  in longer battery runtime.
}

  Please do not re-encode any existing audio files from one lossy format to 
  another based upon the above mentioned. This will reduce the audio quality.
  If you have the choice, select the best suiting codec when encoding the 
  original source material.
}

\opt{swcodec}{
\subsection{Sound settings}
  In general all kinds of sound processing will need more CPU time and therefore
  consume more power. The less sound processing you use, the better it is for 
  the battery runtime (for options see \reference{ref:configure_rockbox_sound}).
}

\input{rockbox_interface/hotkeys.tex}

